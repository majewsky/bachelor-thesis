\documentclass[
 paper=A4,pagesize=automedia,fontsize=12pt,
 BCOR=10mm,
 twoside,headinclude,footinclude=false,
 fleqn,
 bibtotocnumbered,          % show bibl. refs in TOC
 liststotoc,                % show image list in TOC
 listsleft,
 pointlessnumbers,          % no point after heading index
 cleardoublepage=empty      % no page number on vacant pages
]{scrbook}

% encoding
\usepackage[utf8]{inputenc}
\usepackage[T1]{fontenc}
% language
\usepackage[english]{babel}
% fonts
\setkomafont{disposition}{\normalcolor\bfseries} % keine serifenlose Schrift
\usepackage{mathpazo}
% for being able to copy text out of the PDF
\usepackage{cmap}
% layout
\setlength\parindent{0em}
\usepackage{scrpage2} \pagestyle{scrheadings}
                      \clearscrheadfoot
                      \ihead{\headmark}\ohead{\pagemark}
                      \automark[section]{chapter}
                      \setheadsepline{0.5pt}
\usepackage{setspace} \onehalfspacing
\deffootnote{1em}{1em}{\textsuperscript{\thefootnotemark }}
% math mode formatting etc.
\usepackage{amsmath,amsfonts,amssymb,sistyle,xcolor,delim,tikz,calc,lipsum}
\SIstyle{German}
% presentation of floats
\usepackage{flafter,afterpage}
\usepackage[section]{placeins}
\usepackage[margin=8mm,font=small,labelfont=bf,format=plain]{caption}
\usepackage[margin=8mm,font=small,labelfont=bf,format=hang]{subcaption}
% put footnotes below floats
\usepackage[bottom]{footmisc}

\numberwithin{equation}{chapter}
\numberwithin{figure}{chapter}
\numberwithin{table}{chapter}

% hyperref is loaded as the last package (as recommended by its manual)
\usepackage[hyperindex,pdfborder={0 0 0},pdfa]{hyperref}

%%%%%%%%%%%%%%%%%%%%%%%%%%%%%%%%%%%%%%%%%%%%%%%%%%%%%%%%%%%%%%%%%%%%%%%%%%%%%%%%

\delimdef\kla#1{\dleft(#1\dright)}
\delimdef\brk#1{\dleft[#1\dright]}
\delimdef\brc#1{\dleft\{#1\dright\}}
\delimdef\abs#1{\dleft|#1\dright|}
\delimdef\floor#1{\dleft\lfloor#1\dright\rfloor}
\delimdef\ceil#1{\dleft\lceil#1\dright\rceil}
\delimdef\dangle#1{\dleft\langle#1\dright\rangle}
\delimdef\skal#1#2{\dleft\langle#1,#2\dright\rangle}
\delimdef\norm#1{\dleft\|#1\dright\|}

\newcommand\ub{\mathcal B}
\newcommand\ug{\mathcal G}
\newcommand\uh{\mathcal H}
\newcommand\ul{\mathcal L}
\newcommand\um{\mathcal M}
\newcommand\up{\mathcal P}
\newcommand\zn{\mathbb N}
\newcommand\zr{\mathbb R}
\newcommand\zz{\mathbb Z}
\newcommand\argmax{\qopname\relax m{argmax}}
\newcommand\eps{\varepsilon}

%%%%%%%%%%%%%%%%%%%%%%%%%%%%%%%%%%%%%%%%%%%%%%%%%%%%%%%%%%%%%%%%%%%%%%%%%%%%%%%%

\begin{document}

\frontmatter

\begin{titlepage}
 \begin{center}
  \vspace*{5em}

  \begin{singlespace}\bfseries\Huge
   Training of Hidden Markov models as an instance of the expectation maximization algorithm
  \end{singlespace}

  \vspace*{5em}

  \begin{singlespace}\large
   Bachelorarbeit \\ zur Erlangung des Hochschulgrades \\ Bachelor of Science
  \end{singlespace}\medskip

  \vspace*{4em}

  vorgelegt von \\
  {\large Stefan Majewsky} \\
  geboren am 06.11.1989 in Schwerin

  \vspace*{4em}

  \begin{singlespace}\large
   Technische Universität Dresden \\
   Fakultät Informatik \\
   Institut für Theoretische Informatik \\
   Lehrstuhl für Grundlagen der Programmierung
  \end{singlespace}

  \vspace*{4em}

  \begin{singlespace}
   Betreuer: Dipl.-Inf.~Kilian~Gebhardt \\
   Verantwortlicher Hochschullehrer: Prof.~Dr.-Ing.~habil.~Heiko~Vogler
  \end{singlespace}

 \end{center}
\end{titlepage}

\cleardoublepage
\tableofcontents
\mainmatter

\backmatter
\listoffigures % TODO: necessary?
\bibliographystyle{gerunsrt} % TODO: style okay?
\bibliography{citations}

\clearpage
\thispagestyle{empty}
\minisec{Erklärung}\vspace*{1.5em}

Hiermit erkläre ich, dass ich diese Arbeit im Rahmen der Betreuung im Institut
für Theoretische Informatik ohne unzulässige Hilfe Dritter verfasst und alle
Quellen als solche gekennzeichnet habe.

\vspace*{15em}

Stefan Majewsky \par
Dresden, {\color{red}August 2017}

\end{document}
