\documentclass[
 paper=A4,pagesize=automedia,fontsize=11pt,
 BCOR=10mm,DIV=16,
 twoside,headinclude,footinclude=false,
 bibtotocnumbered,          % show bibl. refs in TOC
 liststotoc,                % show image list in TOC
 listsleft,
 pointlessnumbers,          % no point after heading index
 cleardoublepage=empty      % no page number on vacant pages
]{scrbook}

% pdfx wants to be first
\usepackage[a-1b,latxmp]{pdfx}

% encoding
\usepackage[utf8]{inputenc}
\usepackage[T1]{fontenc}
% language
\usepackage[english]{babel}
% fonts
\setkomafont{disposition}{\normalcolor\bfseries} % keine serifenlose Schrift
\usepackage{mathpazo}
% for being able to copy text out of the PDF
\usepackage{cmap}
% layout
\setlength\parindent{0em}
\usepackage{scrpage2} \pagestyle{scrheadings}
                      \clearscrheadfoot
                      \ihead{\headmark}\ohead{\pagemark}
                      \automark[section]{chapter}
                      \setheadsepline{0.5pt}
\usepackage{setspace} \onehalfspacing
\deffootnote{1em}{1em}{\textsuperscript{\thefootnotemark }}
% math mode formatting etc.
\usepackage{amsmath,amsfonts,amssymb,amsthm,sistyle,xcolor,delim}
\SIstyle{German}
\usepackage{graphicx,tikz,calc,tabularx}
\usetikzlibrary{backgrounds,calc}
% presentation of floats
\usepackage{flafter,afterpage}
\usepackage[section]{placeins}
\usepackage[margin=8mm,font=small,labelfont=bf,format=plain]{caption}
\usepackage[margin=8mm,font=small,labelfont=bf,format=hang]{subcaption}
% rendering of algorithms
\usepackage{algorithmic}
\usepackage[chapter]{algorithm}
\newcommand\algorithmheader[1][]{%
 \expandafter\def\expandafter\algorithmicrequire{\begin{minipage}{6em}\textbf{#1}~\end{minipage}}\REQUIRE%
}
\renewcommand\algorithmicrequire{\textbf{Input:}}
\renewcommand\algorithmicensure{\textbf{Output:}}
\newcommand\SUFFIXFOR[1]{\textbf{for}~#1}
% put footnotes below floats
\usepackage[bottom]{footmisc}

\newcounter{theorem}
\theoremstyle{definition}
\newtheorem{theorem}[theorem]{Theorem}
\newtheorem{lemma}[theorem]{Lemma}
\newtheorem{definition*}[theorem]{Definition}

\newenvironment{definition}%
 {\begin{definition*}\pushQED{\qed}}%
 {\popQED\end{definition*}}%

\numberwithin{theorem}{chapter}
\numberwithin{equation}{chapter}
\numberwithin{figure}{chapter}
\numberwithin{table}{chapter}

% hyperref is loaded as the last package (as recommended by its manual)
\hypersetup{hyperindex,pdfborder={0 0 0},pdfa}

%%%%%%%%%%%%%%%%%%%%%%%%%%%%%%%%%%%%%%%%%%%%%%%%%%%%%%%%%%%%%%%%%%%%%%%%%%%%%%%%

\delimdef\kla#1{\dleft(#1\dright)}
\delimdef\brk#1{\dleft[#1\dright]}
\delimdef\brc#1{\dleft\{#1\dright\}}
\delimdef\abs#1{\dleft|#1\dright|}
\delimdef\floor#1{\dleft\lfloor#1\dright\rfloor}
\delimdef\ceil#1{\dleft\lceil#1\dright\rceil}
\delimdef\dangle#1{\dleft\langle#1\dright\rangle}
\delimdef\skal#1#2{\dleft\langle#1,#2\dright\rangle}
\delimdef\norm#1{\dleft\|#1\dright\|}

\usepackage{stmaryrd}
\delimdef\lang#1{\dleft\llbracket#1\dright\rrbracket}
\delimdef\stepmap#1{\llparenthesis#1\rrparenthesis}

\newcommand\ub{\mathcal B}
\newcommand\ug{\mathcal G}
\newcommand\uh{\mathcal H}
\newcommand\ul{\mathcal L}
\newcommand\um{\mathcal M}
\newcommand\up{\mathcal P}
\newcommand\ur{\mathcal R}
\newcommand\zn{\mathbb N}
\newcommand\zr{\mathbb R}
\newcommand\zz{\mathbb Z}
\newcommand\argmax{\qopname\relax m{argmax}}
\newcommand\eps{\varepsilon}
\newcommand\mle{\operatorname{mle}}
\newcommand\cmle{\operatorname{cmle}}
\newcommand\pos{\operatorname{pos}}
\newcommand\osucc{\operatorname{succ}}
\newcommand\rk{\operatorname{rk}}
\newcommand\cnt{\operatorname{count}}

%%%%%%%%%%%%%%%%%%%%%%%%%%%%%%%%%%%%%%%%%%%%%%%%%%%%%%%%%%%%%%%%%%%%%%%%%%%%%%%%

\begin{document}

\frontmatter

\begin{titlepage}
 \begin{center}
  \vspace*{5em}

  \begin{singlespace}\bfseries\Huge
   Training of Hidden Markov models as an instance of the expectation maximization algorithm
  \end{singlespace}

  \vspace*{5em}

  \begin{singlespace}\large
   Bachelorarbeit \\ zur Erlangung des Hochschulgrades \\ Bachelor of Science
  \end{singlespace}\medskip

  \vspace*{4em}

  vorgelegt von \\
  {\large Dipl.-Phys.~Stefan Majewsky} \\
  geboren am 06.11.1989 in Schwerin

  \vspace*{4em}

  \begin{singlespace}\large
   Technische Universität Dresden \\
   Fakultät Informatik \\
   Institut für Theoretische Informatik \\
   Lehrstuhl für Grundlagen der Programmierung
  \end{singlespace}

  \vspace*{4em}

  \begin{singlespace}
   Betreuer: Dipl.-Inf.~Kilian~Gebhardt \\
   Verantwortlicher Hochschullehrer: Prof.~Dr.-Ing.~habil.~Heiko~Vogler
  \end{singlespace}

 \end{center}

 \vspace*{5em}
 \begin{singlespace}
  \hspace{0.18\linewidth}
  \begin{minipage}{0.1\linewidth}
   \includegraphics[width=\linewidth]{img/by-nd.pdf}
  \end{minipage}
  \hspace{0.02\linewidth}
  \begin{minipage}{0.6\linewidth}\footnotesize\flushleft
   This work is licensed under a \href{https://creativecommons.org/licenses/by-nd/4.0/}{Creative Commons Attribution-NoDerivatives 4.0 International License}.
  \end{minipage}\\[-1em]
  \begin{center}\footnotesize
   Source code at: \url{https://github.com/majewsky/bachelor-thesis/}
  \end{center}
 \end{singlespace}
\end{titlepage}

\cleardoublepage
\tableofcontents
\mainmatter

\chapter{Introduction}

In natural language processing, it is frequently necessary to judge the
correctness of sentences generated by some algorithm. For example, a
speech-to-text translator that only recognizes individual words might produce
the following two candidate sentences for the same input audio sample:
\begin{align*}
 \text{He ate soup.} &&
 \text{He aid soup.}
\end{align*}

Both sentences sound the same, but the second candidate sentence should be
discarded because it is syntactically wrong. As another example, a translator
algorithm that translates from another language to English might generate the
following two candidate sentences for some input text:
\begin{align*}
 \text{This is a small red ball.} &&
 \text{This is a red small ball.}
\end{align*}

Both sentences are syntactically correct, but the first sentence should be
preferred because it conforms to the customary rules for adjective ordering in
the English language. Finally, consider a spelling correction process that is
applied to the following sentence: \cite{kukich1992}
\begin{align*}
 \text{The design \textbf{an} construction of the system will take more than a year.}
\end{align*}

Possible replacements for the syntactically wrong word ``an'' include ``a'' and
``and''. In all these situations, a \emph{language model} can be employed to
choose the best result from a set of candidates.
\cite{stolcke2002,youngetal2005} A language model assigns a probability $p_\omega(v)$
to each sentence $v\in\Sigma^*$ (with words from the set $\Sigma$), such that
correct sentences receive a higher probability than incorrect sentences.
\[
 1 > p_\omega\mbig\kla{\text{He ate soup.}} > p_\omega\mbig\kla{\text{He aid soup.}} > p_\omega\mbig\kla{\text{He He He soup.}} > 0
\]

The probability distribution $p_\omega$ is determined by a \emph{model
parameter} $\omega\in\Omega$. The language model defines $\Omega$ and describes
how to obtain $p_\omega$ for any $\omega$. $\omega$ is chosen by a suitable
\emph{training algorithm} using a training corpus $c$ containing known good
sentences, such that $\omega$ maximizes the likelihood of this corpus,
\[
 \prod_{v\in c} p_\omega(v).
\]

\section{N-gram models}

One of the simplest language models is the \emph{bigram model}.
\cite{vogler2015} It interprets the generation of a sentence as a stochastic
process, wherein each word is chosen with a probability conditional on the word
that appeared before it:
\[
 p(\text{He ate soup.}) = b(\text{He}|\#) \cdot b(\text{ate}|\text{He}) \cdot b(\text{soup}|\text{ate}) \cdot b(\#|\text{soup}),
\]
where the model parameter $b$ is a conditional probability distribution of
$\Sigma\cup\brc{\#}$ given $\Sigma\cup\brc{\#}$ and $\#$ is a placeholder word
that stands in for the start and end of the sentence. More generally,
\[
 p(v = v_1\cdots v_n) = b(v_1|\#) \cdot \prod_{i=2}^n b(v_i|v_{i-1}) \cdot b(\#|v_n).
\]
The model parameter $b$ can be chosen by a very simple training algorithm
\cite[pp.~123]{jm09}: Across the corpus, count all bigrams (i.e., all pairs of
words occurring one directly after the other) in the corpus, and also count how
often each word occurs at the start and at the end of a sentence, respectively.
Then set the bigram probability $b$ to these counts, normalized such that
$\sum_{v'} b(v|v')=1$ for every $v\in V$.

The bigram model can be generalized to the \emph{N-gram model}, wherein the
next word is chosen with a probability conditional on the $n-1$ words before
it. Training then counts $n$-grams, i.~e.~sequences of $n$ words, hence the
name \emph{N-gram model}. The bigram model is recovered for $n=2$.

Besides the already mentioned applications where language models augment
translators, N-gram models are useful in \emph{augmentive communication}:
Virtual keyboards on smartphones and tablets predict the next word by looking
at the previous $n$ words, thus improving typing speed and accuracy.
\cite{hasan2004n} A similar assistive word prediction system is part of speech
synthesis programs used by disabled persons such as the physicist Stephen
Hawking. \cite{newelletal1998}

\section{Hidden Markov model}

\begin{figure}[t!]
 \label{fig:language-models}
 \caption{Graphic representation of example instances of language models. Left:
 Bigram model, where arrows $v\to v'$ represent a bigram probability
 $b(v'|v)\neq0$.  The start/end marker $\#$ is shown twice to make the diagram
 more readable. Right: Hidden Markov model, with the hidden states ``noun'' and
 ``verb''; adapted from \cite{nel13}. Arrows $q\to q'$ represent
 a transition probability $t(q'|q)\neq0$. Dashed arrows $q\dashrightarrow v$ represent an
 emission probability $e(v|q)\neq0$.}
 \begin{align*}
  \begin{tikzpicture}[baseline=(current bounding box.north)]
   \begin{scope}[every node/.style={rectangle,draw,text height=1.0em,inner sep=1ex,anchor=north}]
    \node (v0) at (1.5,2) {\#};
    \node (v11) at (0,0) {He};
    \node (v12) at (3,0) {She};
    \node (v21) at (0,-2) {ate};
    \node (v22) at (3,-2) {drank};
    \node (v31) at (0,-4) {soup};
    \node (v32) at (3,-4) {tea};
    \node (v4) at (1.5,-6) {\#};
   \end{scope}
   \begin{scope}[every node/.style={rectangle,inner sep=0.2ex}]
    \draw[->] (v0) -- node[auto,swap] {$0.5$} (v11);
    \draw[->] (v0) -- node[auto] {$0.5$} (v12);
    \draw[->] (v11) -- node[auto,swap] {$0.5$} (v21);
    \draw[->] (v12) -- node[auto,pos=0.2] {$0.5$} (v21);
    \draw[->] (v11) -- node[auto,pos=0.2,swap] {$0.5$} (v22);
    \draw[->] (v12) -- node[auto] {$0.5$} (v22);
    \draw[->] (v21) -- node[auto,swap] {$1$} (v31);
    \draw[->] (v22) -- node[auto] {$0.3$} (v31);
    \draw[->] (v22) -- node[auto] {$0.7$} (v32);
    \draw[->] (v31) -- node[auto] {$1$} (v4);
    \draw[->] (v32) -- node[auto] {$1$} (v4);
   \end{scope}
  \end{tikzpicture}
  &&
  \begin{tikzpicture}[baseline=(current bounding box.north)]
   \begin{scope}[every node/.style={circle,draw}]
    \node (q0) at (0,6) {\#};
    \node (q1) at (0,3) {noun};
    \node (q2) at (0,0) {verb};
   \end{scope}
   \begin{scope}[every node/.style={anchor=west}]
    \node (v11) at (4,5) {He};
    \node (v12) at (4,4) {She};
    \node (v13) at (4,3) {soup};
    \node (v14) at (4,2) {tea};
    \node (v21) at (4,0) {drank};
    \node (v22) at (4,-1) {ate};
   \end{scope}
   \begin{scope}[every node/.style={rectangle,inner sep=0.3ex}]
    \draw[->] (q0) edge[bend right=20] node[auto,swap] {$1$} (q1);
    \draw[->] (q1) edge[bend right=20] node[auto,swap] {$0.5$} (q2);
    \draw[->] (q1) edge[bend right=20] node[auto,swap] {$0.5$} (q0);
    \draw[->] (q2) edge[bend right=20] node[auto,swap] {$1$} (q1);
    \draw[dashed,->] (q1) edge[bend left=20] node[auto,pos=0.6] {$0.3$} (v11);
    \draw[dashed,->] (q1) edge[bend left=10] node[auto,pos=0.6] {$0.3$} (v12);
    \draw[dashed,->] (q1) edge node[auto] {$0.25$} (v13);
    \draw[dashed,->] (q1) edge[bend right=10] node[auto,pos=0.6,swap] {$0.15$} (v14);
    \draw[dashed,->] (q2) edge node[auto] {$0.5$} (v21);
    \draw[dashed,->] (q2) edge node[auto,pos=0.6,swap] {$0.5$} (v22);
   \end{scope}
  \end{tikzpicture}
 \end{align*}
\end{figure}

A further generalization of N-gram models leads to the \emph{Hidden Markov
model}. In this model, the emission probability of a word depends not on the
previous words, but on the progression of a state machine that is not visible
from the outside.  Every time a word needs to be emitted, the state machine
progresses to a new state according to a transition probability distribution
$t$ dependent on the previous state, and the next word is predicted by an
emission probability distribution $e$ dependent on the new state. The symbol
$\#$ is used as the start and end state of the state machine.
For example, if the state sequence that generated the sentence ``He ate soup''
was ``noun-verb-noun'', then the probability of that sentence would be
\begin{align*}
 p
  =&\; t(\text{noun}|\#) \cdot e(\text{He}|\text{noun}) \cdot t(\text{verb}|\text{noun}) \cdot e(\text{ate}|\text{verb}) \\
  &\cdot t(\text{noun}|\text{verb}) \cdot e(\text{soup}|\text{noun}) \cdot t(\#|\text{noun}).
\end{align*}

Since the state sequence is typically not known, the sum over all possible
state sequences must be computed. Therefore,
\[
 p(v=v_1\cdots v_n) = \sum_{q_1,\ldots,q_n} t(q_1|\#) \cdot e(v_1|q_1) \cdot \prod_{i=2}^n \mbig\brk{t(q_i|q_{i-1}) \cdot e(v_i|q_i)} \cdot t(\#|q_n).
\]

N-gram models can be interpreted a special case of the Hidden Markov model, by
using the state space $Q = V^n$ and defining the transmission and emission
probability in terms of the N-gram probability. For example, for bigrams:
\begin{align*}
 t(v_1v_2|v_1'v_2') &:= \begin{cases}
  b(v_2|v_1) & \text{if } v_1 = v_2', \\
  0 &\text{otherwise},
 \end{cases} &
 e(v|v_1v_2) &:= \begin{cases}
  1 & \text{if } v = v_2, \\
  0 &\text{otherwise}.
 \end{cases}
\end{align*}

Training for the Hidden Markov model is not as straight-forward as for the
bigram model, since the training corpus typically only contains the observed
sentences, not the state sequences that produced them. The standard training
algorithm for Hidden Markov models is the \emph{Baum-Welch algorithm}.
\cite{baupetsouwei70,baum1972}

\section{Outlook}

Many training algorithms for language models are instances of the
\emph{expectation-maximization (EM) algorithm}. \cite{demlairub77} Chapter 2
will introduce a general framework for EM algorithms that first appeared in
\cite{bucstuvog15}.

Following the short motivation above, chapter 3 will define Hidden Markov
models more formally, and discuss algorithms that act on them. Chapter 4 will
specifically focus on the Baum-Welch algorithm, and show that it is an instance
of the general EM algorithms laid out in chapter 2.

\chapter{Expectation-maximization algorithms}

The generic framework for expectation-maximization (EM) algorithms introduced
by \cite{bucstuvog15} applies to both language models and translation models.
This chapter will introduce terminology and notation from \cite{bucstuvog15} as
far as is necessary to apply this framework to the Hidden Markov Model in the
subsequent chapters.

\section{Preliminaries}

The set $\brc{0,1,2,\ldots}$ of non-negative integers and the set of
non-negative reals shall be denoted by $\zn$ and $\zr_{\geq0}$, respectively.
We assume that
\begin{align*}
 0^0 &:= 1, &
 \log 0 &:= -\infty, &
 0 (-\infty) &= \log 0^0 = 1.
\end{align*}

\begin{definition}
 Given a countable set $X$, a mapping $c: X \to \zr_{\geq0}$ is called a
 \emph{$X$-corpus}.
\end{definition}

When used as input for a language model's training algorithm, $X$ is the set of
all sentences consisting of words from the language in question, and $c(x)$
describes how often a sentence $x\in X$ occurs in the corpus $c$. The
\emph{size} of the corpus is defined as
\begin{equation*}
 \abs c := \sum_{x\in X} c(x).
\end{equation*}

\begin{definition}
 A \emph{probability distribution of $X$} is an $X$-corpus of size 1.
\end{definition}

The set of all such probability distributions is denoted by $\um(X)$.
Probability distributions can be derived from corpora:

\begin{definition}
 Given a non-empty and finite $X$-corpus $c$, the \emph{empirical probability
 distribution} $\tilde c$ is defined as
 \[
  \tilde c(x) = \frac{c(x)}{\abs c}.
 \]
\end{definition}

This expression is not well-defined for $\abs c = 0$ or $\abs c = \infty$,
hence the requirement for $c$ to be non-empty and finite. If this cannot be
guaranteed, a fall-back function can be added.

\begin{definition}
 Given $p\in\um(X)$, a \emph{normalization mapping with fall-back $p$} is the
 mapping
 \[
  \overline p: \zr_{\geq0}^X \to \um(X),
  \quad
  c \mapsto \begin{cases}
   \tilde c & \text{if } 0 < \abs c < \infty, \\ p & \text{otherwise}.
  \end{cases}
 \]
\end{definition}

\begin{definition}
 Given an $X$-corpus $c$ and $p\in\um(X)$, the \emph{likelihood of $c$ under
 $p$} is
 \[
  p(c) := \prod_{x\in X} p(x)^{c(x)}.
 \]
\end{definition}

The likelihood describes the probability of observing the sentences from the
corpus $c$ when sentences occur with the probability distribution described by
$p$. A training algorithm will take $c$ as an input, and seek to find an
admissible $p$ such that $p(c)$ is maximized. If any $p$ is admissible, then
for non-empty and finite $c$, the optimal choice is $p = \tilde c$:

\begin{lemma}\label{lemma:empirical1}
 Let $c$ be a non-empty and finite $X$-corpus. Then $\tilde c(c) \geq p(c)$ for
 every $p\in\um(X)$.
\end{lemma}

\begin{proof}
 Since $\log$ is monotone, it suffices to show that $\log\tilde c(c) \geq \log
 p(c)$. Using Gibbs' inequality,
 \begin{align*}
  \log \tilde c(c)
  &= \sum_{x\in X} c(x) \cdot \log \tilde c(x)
  = \abs c \cdot \sum_{x\in X} \tilde c(x) \cdot \log \tilde c(x) \\
  &\geq \abs c \cdot \sum_{x\in X} \tilde c(x) \cdot \log p(x)
  = \sum_{x\in X} c(x) \cdot \log p(x)
  = p(c)
  \qedhere
 \end{align*}
\end{proof}

However, using $p = \tilde c$ directly is not useful because this probability
distribution is grossly overfitted: It will assign zero probability to any
sentence not in the original corpus. A useful language model thus limits the
set of admissible $p$ by describing the probability distribution in terms of
\emph{model parameters} $\omega\in\Omega$.

\begin{definition}
 Given a set $\Omega$, an \emph{$\Omega$-probability model for $X$} is a
 mapping $p:\Omega\to\um(X)$.
\end{definition}

Instead of $p(\omega)$, we write $p_\omega$. Training shall then find
$\omega\in\Omega$ such that $p_\omega(c)$ is maximized.

\begin{definition}
 Given a set $\Omega$ and an $\Omega$-probability model $p$ for $X$, the
 \emph{maximum likelihood estimator} for $p$ is the mapping
 \[
  \mle_p: \zr_{\geq0}^X \to \up(\Omega),
  \quad
  c \mapsto \argmax_\omega p_\omega(c).
 \]
\end{definition}

$\mle_p(c)$ is the set of all $\omega$ with maximal likelihood, but training
only needs to find a single $\hat\omega \in \mle_p(c)$. Computing $\mle_p(c)$
by brute force is typically not tractible because the set $\Omega$ is infinite
(usually countably infinite). However, there is one easily solvable special
case.

\begin{lemma}\label{lemma:empirical2}
 Let $c$ be a finite $X$-corpus, and $p$ a $\Omega$-probability model for $X$.
 If there exists $\hat\omega\in\Omega$ such that $p_{\hat\omega} = \tilde c$,
 then $\hat\omega\in\mle_p(c)$.
\end{lemma}

\begin{proof}
 If $c$ is empty, then $p_\omega(c) = 1$ for every $\omega$, and thus
 $\mle_p(c) = \Omega \ni \hat\omega$. Otherwise, by
 Lemma~\ref{lemma:empirical1}, $p_{\hat\omega}(c) = \tilde c(c) \geq
 p_\omega(c)$ for every $\omega\in\Omega$, and thus $\hat\omega \in \mle_p(c)$.
\end{proof}

\section{Algorithmic skeleton}

\begin{algorithm}[t]
 \caption{Algorithmic skeleton for EM of language models according to \cite{bucstuvog15}}
 \label{alg:skeleton}
 \begin{algorithmic}[1]
  \algorithmheader[Input:] $X$-corpus $c$
  \algorithmheader         $\Omega$-probability model $p$ for $Y\times X$
  \algorithmheader         some initial parameter $\omega_0 \in \Omega_0$ where $\Omega_0 := \brc{\omega\in\Omega: p_\omega(c) \neq 0}$
  \algorithmheader[Implicit:] step mapping $\psi:\Omega_0\to\up(\Omega)$
  \algorithmheader            \hspace{1em} such that $\omega'\in\psi(\omega)$ implies $p_\omega(c) \leq p_{\omega'}(c)$
  \algorithmheader[Output:] sequence $\omega_1,\omega_2,\ldots\in\Omega_0$
  \algorithmheader            \hspace{1em} such that $p_{\omega_0}(c), p_{\omega_1}(c), p_{\omega_2}(c)$ nondecreasing

  \STATE $i\leftarrow 0$
  \WHILE{not converged}
   \STATE $\omega_{i+1} \leftarrow \text{select a member of $\psi(\omega_i)$}$
   \STATE output $\omega_{i+1}$
   \STATE $i\leftarrow i+1$
  \ENDWHILE
 \end{algorithmic}
\end{algorithm}

The major complication that occurs when trying to train a typical language
model is that not all required data is present in the corpus. For example, a
probabilistic context-free grammar is described by the probability distribution
of derivation rules. \cite{laryou90} When the training data consists of full
parse trees (\emph{supervized training}), the optimal probability distribution
can be found by simply counting how many times each rule is used across all
these parse trees, and then computing the empirical probability distribution
for this corpus. Most of the times, however, the training data will consist
only of sentences (\emph{unsupervized training}). The information about how to
parse the sentences is hidden.

The same problem arises with the Hidden Markov Model: When training data is not
already annotated with state information, the information which states
correspond to which words from the training data remains hidden. Expectation
maximization algorithms can be used when parts of the training data are hidden
in such a way.

For the remainder, let
\begin{itemize}\setlength\itemsep{-0.3em}
 \item $X$ and $Y$ be countable sets,
 \item $\Omega$ be a set,
 \item $c$ be a finite $X$-corpus and
 \item $p$ be a $\Omega$-probability model for $Y\times X$.
\end{itemize}

$c$ represents the set of training data. Each $x\in\operatorname{supp}(c)$ is
an \emph{observation}. To judge its probability under a $p_\omega$, additional
\emph{hidden information} $y\in Y$ is required. For notational convenience, we define
\begin{align*}
 p_\omega(c) &:= \prod_{x\in X} p_\omega(x)^{c(x)}, &
 \text{where } p_\omega(x) &:= \sum_{y\in Y} p_\omega(x,y).
\end{align*}

That is, even though $p_\omega$ is a probability distribution over $Y\times X$,
we allow to take the likelihood of the $X$-corpus $c$ under $p$ by aggregating
the probabilities for all hidden information $y$ that lead to a certain
observation $x$.

The basic pattern for expectation maximization is outlined in
algorithm~\ref{alg:skeleton}. The algorithm starts with an initial $\omega_0$
such that $p_{\omega_0}(c) \neq 0$. It then iteratively employs a step mapping
to choose the next $\omega_i$ with a higher (or at least equal) likelihood than
the one that came before.

\begin{definition}
 A \emph{step mapping} is a mapping $\psi:\Omega_0\to\up(\Omega)$ which is
 nondecreasing in the following manner:
 \[
  \forall \omega\in\Omega_0: \forall \omega'\in\psi(\Omega): p_\omega(c) \leq p_{\omega'}(c)
 \]
\end{definition}

The step mappings that we will consider will typically consist of two steps:
\begin{enumerate}
 \item \emph{Expectation:} The training data $c$ is converted into a
  \emph{complete-data corpus}. Using the $\omega_i$ from the previous
  iteration, the complete-data corpus estimates how hidden information
  contributes to the observations in the original corpus.
 \item \emph{Maximization:} A suitable maximum-likelihood estimator is applied
  to the complete-data corpus to choose $\omega_{i+1}$.
\end{enumerate}

%TODO: this section may use some citations because it makes claims
This back and forth of using the current $\omega$ to enrich the training data
and using the enriched data to find a better $\omega$ will converge towards a
local maximum of likelihood. The iteration is therefore usually aborted after
the desired running time has been exceeded, or after the changes of
$p_{\omega_i}(c)$ per iteration have become smaller than some threshold.

\cite{bucstuvog15} identify three types of step mappings that build on each
other, each one more specific than the one before it. Since the training of
Hidden Markov Models will be identified as an instance of the most specific
step mapping, the remainder of this chapter will introduce all three in order.

\section{Corpus-based step mapping}

The most general type of complete-data corpus can be obtained by distributing
$c(x)$ among the hidden information $y$ according to the probability
distribution $p_\omega$:
\[
 c\!\dangle{\omega,p}(y,x) := \begin{cases}
  c(x) \cdot \frac{p_\omega(y,x)}{p_\omega(x)} & \text{if } p_\omega(x) \neq 0, \\
  0 & \text{if } p_\omega(x) = 0.
 \end{cases}
\]
Recall that $p_\omega(x) = \sum_y p_\omega(y,x)$. Therefore,
$\abs{c\!\dangle{\omega,p}} = \abs c$. This corpus now has the correct
structure for plugging it into $\mle_p$, yielding the \emph{corpus-based step
mapping}\footnote{The proof that this step mapping is nondecreasing can be
found in \cite[pp.~10]{bucstuvog15}.}
\[
 \stepmap p_\mathrm{cb}: \Omega_0\to\up(\Omega),
 \quad
 \omega \mapsto \mle_p\mbig\kla{c\!\dangle{\omega,p}} = \argmax_{\omega'} p_{\omega'}\mbig\kla{c\!\dangle{\omega,p}}.
\]

For very simple language models, the $\argmax$ can be solved at this point
already. To apply lemma~\ref{lemma:empirical2}, $\hat\omega$ needs to be found
such that $p_{\hat\omega} = \widetilde{c\!\dangle{\omega,p}}$. This operation
is typically hard, which is why a more specific step mapping is helpful.

\section{Simple counting step mapping}

The next such step mapping requires the language model to be described by a
\emph{counting information}. Before defining this term, some additional
notation needs to be introduced.

\begin{definition}
 Let $A$ and $B$ be sets. A mapping $p: B \to \um(A)$ is called
 \emph{conditional probability distributions of $A$ given $B$}. The set of all
 such mappings is denoted by $\um(A|B)$.
\end{definition}

To simply notation, we define $p(a|b) := p(b)(a)$.

\begin{definition}
 Let $C\subseteq A\times B$ be a set. Then
 \[
  \um_C(A|B) := \brc{p\in\um(A|B): \operatorname{supp}(p)\subseteq C}
 \]
 is the set of all conditional probability distributions of $A$ given $B$
 constrained to $C$.
\end{definition}

\begin{definition}
 Given a set $\Omega$, a \emph{conditional $\Omega$-probability model for $A$
 and $B$ (constrained to $C$)} is a mapping $q:\Omega\to\um(A|B)$ (or
 $q:\Omega\to\um_C(A|B)$).
\end{definition}

When talking about counting informations (and inside-outside informations in
the next section), we assume the previously established requirements for $X$,
$Y$, $\Omega$, $c$ and $p$. Furthermore, we require that $X$ and $Y$ both
contain a special symbol $\bot$ such that $c(\bot) = 0$. $\bot$ can easily be
added to any previously defined $X$ and $Y$ without affecting the requirement
for countability. The notation $U_{\not\bot} := U\setminus\brc\bot$ shall be
defined for any set $U$.

\begin{definition}
 Let $A$, $B$ and $C$ be sets such that $C\subseteq A\times B$. A
 \emph{counting information} is a triple $\varkappa = (q,\lambda,\pi)$ such
 that
 \begin{align*}
  q &: \Omega\to\um_C(A|B), &
  \lambda &: X_{\not\bot} \times Y_{\not\bot} \to [0,1], &
  \pi &: X_{\not\bot} \times Y_{\not\bot} \to \zr_{\geq0}^C.
 \end{align*}
\end{definition}

The motivation for this definition is to model hidden information $y$ as
consisting of countable events $c\in C$. $\lambda(x,y)$ describes whether (and,
possibly, with what probability) a certain $y$ can be the cause for a certain
observation $x$.\footnote{Most actual instances of $\lambda$ use only integer
images, i.~e.~$\lambda(X_{\not\bot}\times Y_{\not\bot}) = \brc{0,1}$, thus
following this intuitive notion. However, the possibility of using fractional
values for $\lambda(x,y)$ is occasionally useful, e.~g.~to define a counting
information for the IBM Model 1 in \cite[pp.~23]{bucstuvog15}.} For
$\lambda(x,y)>0$, $\pi(x,y)$ is a $C$-corpus that describes how often each
countable event occurs in this hidden information.

Following the assumption that the countable events $C$ fully encode the hidden
information $Y$, we can use these intuitive notions to describe the original
probability model $p$ in terms of the counting information.

\begin{definition}
 Given $\varkappa=(q,\lambda,\pi)$, the \emph{induced model}
 $\varkappa^\flat:\Omega\to\zr_{\geq0}^{Y\times X}$ is given by
 \[
  (\varkappa^\flat)_\omega(y,x) := \begin{cases}
   \lambda(x,y) \cdot q_\omega\mbig\kla{\pi(x,y)} & \text{if } x,y\neq\bot, \\
   1 - \sum_{x',y'\neq\bot} \lambda(x',y') \cdot q_\omega\mbig\kla{\pi(x',y')} & \text{if } x = y = \bot, \\
   0 & \text{otherwise}.
  \end{cases}
 \]
\end{definition}

This definition shows why the introduction of $\bot$ into $X$ and $Y$ was
useful. By defining $(\varkappa^\flat)_\omega(\bot,\bot)$ as above, we ensure
$\mnorm\abs{(\varkappa^\flat)_\omega} = 1$. Therefore, $\varkappa^\flat$ is an
$\Omega$-probability model for $Y\times X$ iff
$(\varkappa^\flat)_\omega(\bot,\bot) \geq 0$. We call $\varkappa$ \emph{proper}
in this case.

Since we now have a probability model $p = \varkappa^\flat$ as required by the
corpus-based step mapping, we can lift its complete-data corpus into the domain
of the counting information, obtaining a new complete-data corpus
\[
 c\!\dangle{\omega,\varkappa}: C \to \zr_{\geq0}, \quad
 (a,b) \mapsto \sum_{x,y} c\mnorm\dangle{\omega,\varkappa^\flat}(y,x) \cdot \pi(x,y)(a,b).
\]

We can now apply a maximum-likelihood estimator for $q$ to arrive at the step
mapping for this class of language models.

\begin{definition}
 Given a set $\Omega$ and a conditional $\Omega$-probability model $q$ for $A$
 and $B$, the \emph{conditional maximum likelihood estimator} for $q$ is the mapping
 \[
  \cmle_q: \zr_{\geq0}^{A\times B} \to \up(\Omega),
  \quad
  c \mapsto \argmax_\omega q_\omega(c).
 \]
\end{definition}

Using this definition, the \emph{simple counting step mapping}\footnote{The
proof that this step mapping is equivalent to
$\mnorm\stepmap{\varkappa^\flat}_\mathrm{cb}$, and thus also nondecreasing, can
be found in \cite[p.~13]{bucstuvog15}.} is
\[
 \stepmap\varkappa_\mathrm{sc}: \Omega_0 \to \up(\Omega), \quad
 \omega \to \cmle_q\mbig\kla{c\mnorm\dangle{\omega,\varkappa}} = \argmax_{\omega'} q_{\omega'}\mbig\kla{c\mnorm\dangle{\omega,\varkappa}}.
\]
The simple counting step mapping has two advantages over
$\stepmap\cdot_\mathrm{cb}$: First, many language models can be described in
terms of the countable events only, such that $\Omega = C$. In this case,
lemma~\ref{lemma:empirical2} can be used to solve the $\argmax$ by simply
computing the empirical probability distribution of
$c\!\dangle{\omega,\varkappa}$.

Second, even if this is not possible, the set of countable events $C$ is
usually much smaller than the set of all observations $X$ or hidden information
$Y$, making the evaluation of the $\argmax$ more tractable than for the
corpus-based step mapping. For example, when considering probabilistic
context-free grammars, $X$ (the set of all sentences) and $Y$ (the set of all
parse trees) are both countably infinite, but $C$ (the set of all derivation
rules) is finite.

\section{Regular tree grammars}

The third and most specific type of step mapping applies to language models
whose hidden information can be described as trees, such that the countable
events are labels in these trees' nodes. We therefore need to define a set of
terms regarding trees and tree grammars first.

\begin{definition}
 An \emph{alphabet} is a finite set.
\end{definition}

\begin{definition}
 Let $\Sigma$ be an alphabet and $V$ be a set. The \emph{set $U_\Sigma(V)$ of
 unranked trees over $\Sigma$ indexed by $V$} is the smallest set $T$ such that
 \[
  V \subseteq T \quad\wedge\quad \forall k\in\zn: \forall \sigma\in\Sigma, t_1,\ldots,t_k\in T: \sigma(t_1,\ldots,t_k)\in T.
 \]
\end{definition}

The notation $\sigma(t_1,\ldots,t_k)$ refers to the tree which has the label
$\sigma$ at its root and the subtrees $t_1,\ldots,t_k$ in that order. For
$k=0$, we shorten $\sigma() \equiv \sigma$. For $V=\emptyset$, we shorten
$U_\Sigma(\emptyset) \equiv U_\Sigma$. Trees from $U_\Sigma(V)$ have labels
from $\Sigma\cup V$ at each node, but labels from $V$ are only permitted at
leafs. For example, for $\Sigma = \brc{\sigma_1,\sigma_2}$ and
$V=\brc{v_1,v_2}$:

\[\begin{matrix}
 \begin{tikzpicture}[every node/.style={rectangle,draw,minimum size=2em},baseline=(current bounding box.center)]
  \node (n)   at (+0,+0.0) { $\sigma_1$ };
  \node (n1)  at (-1,-1.5) { $\sigma_2$ } edge (n);
  \node (n2)  at (+1,-1.5) { $\sigma_2$ } edge (n);
  \node (n11) at (-2,-3.0) { $v_1$ } edge (n1);
  \node (n12) at (+0,-3.0) { $\sigma_1$ } edge (n1);
  \node (n21) at (+2,-3.0) { $v_2$ } edge (n2);
 \end{tikzpicture}
 &\hspace*{4em}&
 \begin{tikzpicture}[every node/.style={rectangle,draw,minimum size=2em},baseline=(current bounding box.center)]
  \node (n)   at (+0,+0.0) { $\sigma_1$ };
  \node (n1)  at (-1,-1.5) { $\sigma_2$ } edge (n);
  \node (n2)  at (+1,-1.5) { $v_1$ } edge (n);
  \node (n11) at (-2,-3.0) { $v_1$ } edge (n1);
  \node (n12) at (+0,-3.0) { $\sigma_1$ } edge (n1);
  \node (n21) at (+2,-3.0) { $v_2$ } edge (n2);
  \draw (n2) circle (0.8);
  \node[draw=none] at (+1.75,-0.75) {!};
 \end{tikzpicture}
 \\\\
 \sigma_1\mbig\kla{\sigma_2(v_1,\sigma_1), \sigma_2(v_2)} \in U_\Sigma(V) &&
 \sigma_1\mbig\kla{\sigma_2(v_1,\sigma_1), v_1(v_2)} \notin U_\Sigma(V)
\end{matrix}\]

We refer to positions in the tree using \emph{Gorn notation}. As a quick
illustration, in the figure below, each node of the left tree is labeled with
its position in the tree:

\vspace*{-1em}\[
 \hspace*{3em}
 \begin{tikzpicture}[every node/.style={rectangle,draw,minimum size=2em},baseline=(current bounding box.center)]
  \node (n)   at (+0,+0.0) { $\eps$ };
  \node (n1)  at (-1,-1.5) { $1$ } edge (n);
  \node (n2)  at (+1,-1.5) { $2$ } edge (n);
  \node (n11) at (-2,-3.0) { $1\ 1$ } edge (n1);
  \node (n12) at (+0,-3.0) { $1\ 2$ } edge (n1);
  \node (n21) at (+2,-3.0) { $2\ 1$ } edge (n2);
 \end{tikzpicture}
 \hspace*{6em}
 \begin{tikzpicture}[baseline=(current bounding box.center)]
  \begin{scope}[every node/.style={rectangle,draw,fill=white,minimum size=2em}]
   \node (n)   at (+0,+0.0) { $\sigma_1$ };
   \node (n1)  at (-1,-1.5) { $\sigma_2$ } edge (n);
   \node (n2)  at (+1,-1.5) { $\sigma_2$ } edge (n);
   \node (n11) at (-2,-3.0) { $v_1$ } edge (n1);
   \node (n12) at (+0,-3.0) { $\sigma_1$ } edge (n1);
   \node (n21) at (+2,-3.0) {} edge (n2);
  \end{scope}
  \begin{scope}[every node/.style={inner sep=0},every pin/.style={inner sep=0},pin distance=1cm]
   \node[draw=none,pin=60:{$t(2\ 1)$}] at (n21.center) { $v_2$ };
  \end{scope}
  \begin{scope}[on background layer]
   \draw[line join=round,draw=black!10,fill=black!10,line width=1.5cm] (n1.center) -- (n11.center) -- (n12.center) -- cycle;
  \end{scope}
  \node at (-2.4,-1.7) { $t|_1$ };
 \end{tikzpicture}
\]

$\eps$ is the empty word. The set of all positions in a tree $t$ shall be
denoted by $\pos(t)$. The right tree illustrates some notation that we're going
to use for trees. For any trees $t,t'$ and position $w\in\pos(t)$,
\begin{itemize}\setlength\itemsep{-0.3em}
 \item $t(w)$ is the label at $w$,
 \item $t|_w$ is the subtree at $w$ (hence, especially, $t|_\eps = t$) and
 \item $t[t']_w$ is the tree that results from replacing $t|_w$ by $t'$.
\end{itemize}

\begin{definition}
 Let $\Sigma$ be an alphabet and $\square\notin\Sigma$. A \emph{1-context over
 $\Sigma$} is a tree $t\in U_\Sigma(\brc\square)$ such that $\square$ appears
 at exactly one position in $t$. The set of all such 1-contexts is denoted by
 $C_\Sigma$.
\end{definition}

1-contexts are used to describe trees that are not fully known yet: The
placeholder symbol $\sigma$ stands in for a missing subtree. Given a 1-context
$t\in C_\Sigma$ and a tree $t'\in U_\Sigma(V)$, we abbreviate $t[t'] :=
t[t']_w$ such that $t(w) = \square$. In other words, $t[t']$ is the tree that
results from $t$ when the $\square$ node is replaced by $t'$.

\begin{definition}
 A \emph{ranked alphabet} is a pair $(R,\rk)$ where $R$ is an alphabet and
 $\rk:R\to\zn$ is a mapping. $\rk$ is said to assign a \emph{rank} or
 \emph{arity} or each symbol in $R$.
\end{definition}

A ranked alphabet is usually denoted only by $R$. The existence of a suitable
$\rk$ is implied. A symbol $\rho\in R$ may be denoted as $\rho^{(k)}$ where
$\rk(\rho) = k$, to inform the reader of the symbol's rank.

\begin{definition}
 Let $R$ be a ranked alphabet and $V$ be a set. The \emph{set $T_R(V)$ of
 ranked trees over $R$ indexed by $V$} is defined by
 \[
  T_R(V) := \mbig\brc{t \in U_R(V) \dmiddle| \forall w\in\pos(t): \rk\mbig\kla{t(w)} = \text{number of children of $w$ in $t$}}.
 \]
\end{definition}

In other words, the number of children of each node in $t$ must be equal to the
rank of its label. For nodes labeled with symbols from $V$, a rank of 0 is
implied.

\begin{definition}
 Let $\Sigma$ be an alphabet. A \emph{regular tree grammar (RTG) over $\Sigma$} is a
 triple $\ug = (Q,q_0,R)$ where
 \begin{itemize}\setlength\itemsep{-0.3em}
  \item $Q$ is a nonempty alphabet (of states),
  \item $q_0\in Q$ is an initial state and
  \item $R\subset Q^*\times\Sigma\times Q$ is a finite ranked alphabet (of
   rules) such that every rule $\mbig\kla{(q_1,\ldots,q_k),\sigma,q}\in R$ has
   rank $k$.
 \end{itemize}
\end{definition}

We will write $\mbig\kla{(q_1,\ldots,q_k),\sigma,q}\in R$ as $q\to\sigma(q_1,\ldots,q_k)$ instead.

\begin{definition}
 Let $\ug=(Q,q_0,R)$ be an RTG over $\Sigma$. The \emph{family
 $\mbig\kla{D^q(\ug)\dmiddle|q\in Q}$ of partial abstract syntax trees oof
 $\ug$} is the smallest $Q$-indexed family $(D^q|q\in Q)$ such that, for all
 $q\in Q$,
 \[
  q \in D^q \quad\wedge\quad \forall \rho=\mbig\kla{q\to\sigma(q_1,\ldots,q_k)}\in R,d_i\in D^{q_i}: \rho(d_1,\ldots,d_k) \in D^q.
 \]
\end{definition}


An RTG describes a language (i.~e.~a countable set) of trees. Trees are derived
by starting with a tree containing only the a root node with the label $q_0$,
then successively replacing $Q$-labeled nodes according to the RTG's rule set
until no more $Q$-labeled nodes are left. The resulting tree is in $U_\Sigma$
and its abstract syntax tree is in $T_R$. Partial abstract syntax trees are in $T_R(Q)$.

For example, consider the RTG $\ug = \mbig\kla{\brc{q_0,q_1},q_0,R}$ where
$\Sigma=\brc{a,b,c}$ and
\[
 R = \brc{ q_0 \to a(q_0,q_1), \quad q_0 \to b, \quad q_1 \to c }.
\]
The tree $a(b,c)$ is derived like this: (Each row shows the partial tree $\in
U_\Sigma(Q)$ on the left and the partial abstract syntax tree $\in T_R(Q)$ on the
right.)
\[\begin{matrix}
 \begin{tikzpicture}[every node/.style={rectangle,draw,minimum size=2em},baseline=(current bounding box.center)]
  \node (n) at (0,0) { $q_0$ };
 \end{tikzpicture}
 &&
 \begin{tikzpicture}[every node/.style={rectangle,draw,minimum size=2em},baseline=(current bounding box.center)]
  \node (n) at (0,0) { $q_0$ };
 \end{tikzpicture}
 \\\\[-1em]
 \downarrow & \text{apply } q_0\to a(q_0,q_1) & \downarrow \\[0.5em]
 \begin{tikzpicture}[every node/.style={rectangle,draw,minimum size=2em},baseline=(current bounding box.center)]
  \node (n) at (0,0) { $a$ };
  \node (n1) at (-1,-1.5) { $q_0$ } edge (n);
  \node (n2) at (+1,-1.5) { $q_1$ } edge (n);
 \end{tikzpicture}
 &\hspace*{4em}&
 \begin{tikzpicture}[every node/.style={rectangle,draw,minimum size=2em},baseline=(current bounding box.center)]
  \node (n) at (0,0) { $q_0 \to a(q_0,q_1)$ };
  \node (n1) at (-1,-1.5) { $q_0$ } edge (n);
  \node (n2) at (+1,-1.5) { $q_1$ } edge (n);
 \end{tikzpicture}
 \\\\[-1em]
 \downarrow & \text{apply } q_0\to b & \downarrow \\[0.5em]
 \begin{tikzpicture}[every node/.style={rectangle,draw,minimum size=2em},baseline=(current bounding box.center)]
  \node (n) at (0,0) { $a$ };
  \node (n1) at (-1,-1.5) { $b$ } edge (n);
  \node (n2) at (+1,-1.5) { $q_1$ } edge (n);
 \end{tikzpicture}
 &\hspace*{4em}&
 \begin{tikzpicture}[every node/.style={rectangle,draw,minimum size=2em},baseline=(current bounding box.center)]
  \node (n) at (0,0) { $q_0 \to a(q_0,q_1)$ };
  \node (n1) at (-1,-1.5) { $q_0\to b$ } edge (n);
  \node (n2) at (+1,-1.5) { $q_1$ } edge (n);
 \end{tikzpicture}
 \\\\[-1em]
 \downarrow & \text{apply } q_1\to c & \downarrow \\[0.5em]
 \begin{tikzpicture}[every node/.style={rectangle,draw,minimum size=2em},baseline=(current bounding box.center)]
  \node (n) at (0,0) { $a$ };
  \node (n1) at (-1,-1.5) { $b$ } edge (n);
  \node (n2) at (+1,-1.5) { $c$ } edge (n);
 \end{tikzpicture}
 &\hspace*{4em}&
 \begin{tikzpicture}[every node/.style={rectangle,draw,minimum size=2em},baseline=(current bounding box.center)]
  \node (n) at (0,0) { $q_0 \to a(q_0,q_1)$ };
  \node (n1) at (-1,-1.5) { $q_0\to b$ } edge (n);
  \node (n2) at (+1,-1.5) { $q_1\to c$ } edge (n);
 \end{tikzpicture}
\end{matrix}\]

From the abstract syntax tree on the right, the tree on the left can be derived easily
by substituting the labels $q = \sigma(q_1,\ldots,q_k)\in R$ by the contained
symbols $\sigma\in\Sigma$. We shall call this projection $\pi_\Sigma: T_R(Q)\to U_\Sigma(Q)$.

\begin{definition}
 Let $\ug$ be an RTG over $\Sigma$. The \emph{language of $\ug$} is the set
 \[
  \lang\ug := \brc{t\in U_\Sigma | \exists d\in D^{q_0}(\ug): t=\pi_\Sigma(d)}.
 \]
 A language $L\subseteq U_\Sigma$ is \emph{recognizable} if there exists an RTG
 $\ug$ over $\Sigma$ such that $\lang\ug = L$.
\end{definition}

A useful property of RTGs is \emph{determinism}: An RTG $\ug$ over $\Sigma$ is
called \emph{deterministic} if, for any $(q_1,\ldots,q_k)\in Q^*$ and
$\sigma\in\Sigma$, there is at most one $q\in Q$ such that $\mbig\kla{q \to
\sigma(q_1,\ldots,q_k)} \in R$. A language is called deterministic if it is
described by a deterministic RTG.

From determinism results unambiguity: An RTG $\ug$ over $\Sigma$ is called
\emph{unambiguous} if, for every tree $t\in\lang\ug$, there exist exactly one
abstract syntax tree $d\in D^{q_0}(\ug)$ such that $\pi_\Sigma(d) = t$. To see why, have
a look at the fully derived tree $t = a(b,c)$ above. Since $\ug$ in this
example is deterministic, the abstract syntax tree can be recovered from $t$ by
traversing the nodes from the bottom up and assigning the rules that are used
at that position. At each position, we know the symbol $\sigma$ at this
position in the tree and the states $q_1,\ldots,q_k$ from the left sides of the
rules used for the child nodes. Therefore, the state $q$ (and therefore the
rule $\rho$) for this position can be chosen deterministically.

\begin{definition}
 A \emph{probabilistic regular tree grammar (PRTG) over $\Sigma$} is a pair
 $(\ug,p)$ of an RTG $\ug=(Q,q_0,R)$ over $\Sigma$ and a mapping
 $p:Q\to\zr_{\geq0}^{Q^*\times\Sigma}$ that is constrained to $R$ in the
 following way:
 \[
  \forall q\in Q, u\in Q^*\times\Sigma: p(q)(u) \neq 0 \Rightarrow (u,q) \in R.
 \]
 $(\ug,p)$ is called \emph{proper} if $p\in\um_R(Q^*\times\Sigma|Q)$. The
 \emph{meaning} of $(\ug,p)$ is the mapping
 \[
  \mbig\lang{(\ug,p)}: U_\Sigma \to \zr_0, \quad
  t \mapsto \sum_{d\in D^{q_0}(\ug): \pi_\Sigma(d)=t} p(d).
 \]
 Herein, $p(d) := p\mbig\kla{\pi(d)}$, where $\pi(d)$ is an $R$-corpus with
 \[
  \pi(d)(p) := \mbig\abs{\mbig\brc{w\in\pos(d): d(w)=p}}.
 \]
\end{definition}

We usually write a PRTG $(\ug,p)$ as just $\ug$ and imply the existence of $p$.
The meaning function assigns a probability to a tree from $\lang\ug$ by summing
the probability of all derivations resulting in that tree, where the
probability of a derivation is the product of the probability of each rule
occurring in it. This implies that rule applications are statistically
independent from each other.

When describing hidden information in terms of PRTG in the next section, the
notion of \emph{inside and outside weights} will be useful to judge, broadly
speaking, how much a certain state contributes to the derivations of a certain
observation.

\begin{definition}
 Let $\ug = (Q,q_0,R)$ be a PRTG over $\Sigma$. The \emph{inside weight} of a
 state $q\in Q$ is given by
 \[
  \beta(q) := \sum_{d\in D^q(\ug)\cap T_R} p(d).
 \]
\end{definition}

The sum goes over all complete abstract syntax trees rooted at $q$. The inside weight
describes the collective probability of all derivations starting at $q$. Inside
weights are usually calculated by noting that each $d\in D^q(\ug)\cap T_R$ must
have a rule of the form $q\to\ldots$ at its root. The remaining probabilities
can then be expressed as the inside weights of the states that emerge this rule
application, giving
\[
 \beta(q) = \sum_{q_1,\ldots,q_k,\sigma} p\mbig\kla{q\to\sigma(q_1,\ldots,q_k)} \cdot \beta(q_1) \cdots \beta(q_k).
\]

The set of all such equations is a non-linear equation system in the $\beta(q)$
for $q\in Q$. In some cases, this system can be solved intuitively by starting
with those equations where no $\beta(q_i)$ occurs on the right side, then
substituting the obtained value in the other equations until they are all
solved. This is not possible, however, if states derive other states in a
cyclic way. For example:
\[
 R = \brc{\begin{aligned}
  \rho_1 &= q_0 \to \sigma_1(q_1, q_1), \\
  \rho_2 &= q_1 \to \sigma_2(q_0), \\
  \rho_3 &= q_1 \to \sigma_3
 \end{aligned}} \quad\leadsto\quad
 \begin{aligned}
  \beta(q_0) &= p(\rho_1) \cdot \beta(q_1)^2 \\
  \beta(q_1) &= p(\rho_2) \cdot \beta(q_0) + p(\rho_3)
 \end{aligned}
\]

In the general case, \cite[pp.~6]{bucstuvog15} shows that $\beta$ is the least fixpoint of the mapping $F: (\zr_{\geq0}\cup\brc\infty)^Q \to (\zr_{\geq0}\cup\brc\infty)^Q$, given by
\[
 F(u)(q) := \sum_{q_1,\ldots,q_k,\sigma} p\mbig\kla{q\to\sigma(q_1,\ldots,q_k)} \cdot u(q_1) \cdots u(q_k).
\]
Therefore,
\[
 \beta = \lim_{n\to\infty} F^n(u_0) \quad\text{where}\quad u_0(q) := 0 \;\forall q\in Q
\]
can be approximated by performing as many iterations of $F$ as desired.

\begin{definition}
 Let $\ug = (Q,q_0,R)$ be a PRTG over $\Sigma$. The \emph{outside weight} of a state $q\in Q$ is given by
 \[
  \alpha(q) := \sum_{d\in D^{q_0}(\ug): \exists d'\in C_R: d = d'[q]} p(d).
 \]
\end{definition}

The sum goes over all partial abstract syntax trees, where only one unexpanded
state is left, and that state is $q$. The outside weight describes the
collective probability of all derivations that use $q$, but without considering
at or below $q$. Similar to what we did with inside weights, we can expand this
definition into
\[
 \alpha(q) = \delta_{q_0}^q + \sum_{q',q_1,\ldots,q_k,\sigma,m:q_m=q} \alpha(q') \cdot p\mbig\kla{q'\to\sigma(q_1,\ldots,q_k)} \cdot \prod_{l\neq m} \beta(q_l)
\]

In this formulation, the sum goes over all rules that derive $q$, that is, $q$
is among the states $q_1,\ldots,q_k$ on the right hand side of the rule, at
index $m$. The outside weight $\alpha(q)$ considers the outside weight
$\alpha(q')$ of the previous state and the inside weight of all states adjacent
to $q$, but not the inside weight of $q$ itself.

Assuming that the inside weights $\beta(q)$ have already been calculated, the
equations for $\alpha(q)$ form a linear equation system that can be solved
efficiently with the standard algorithms for linear equation systems.

\chapter{The Hidden Markov model}

The definitions in this chapter are based on \cite[pp.~210]{jm09}, but use
different variable names in several places to avoid conflicts with the notation
from chapter~2.

\begin{definition}
 A \emph{Hidden Markov model} (HMM) is a quintuple\footnote{This definition
 diverges in structure from \cite{jm09} in one significant way: It uses a
 single state $\#$ instead of a pair of initial state $q_0$ and final state
 $q_F$. This allows us to use a single probability distribution $t$ to describe
 all state transitions, instead of the triple of matrices $(a_{0i})_i$,
 $(a_{ij})_{(i,j)}$, and $(a_{iF})_i$ that appear in \cite{jm09}.} $H =
 (Q,V,\#,t,e)$ such that
 \begin{itemize}\setlength\itemsep{-0.3em}
  \item $Q$ is a non-empty alphabet (of states),
  \item $V$ is a non-empty alphabet (of words),
  \item $\#\notin Q\cup V$ is a separate (initial and final) state,
  \item $t\in\um(Q\cup\brc{\#}|Q\cup\brc{\#})$ and $e\in\um(V|Q)$. \qedhere
 \end{itemize}
\end{definition}

From here on, we will abbreviate $Q\cup\brc{\#}$ as $Q_\#$.

The Hidden Markov model describes a sentence as being the result of the
progression of a probabilistic state machine that starts out in $\#$, traverses
states from $Q$, and in the end reaches $\#$ again. Each time a state from $Q$
is reached, a word from $V$ is emitted. The sequence of all these emitted words
is the sentence that is observed.

When a sentence $v = v_1\cdots v_n\in V^+$ is observed, is must have been
caused by a certain sequence of states $q = q_1\cdots q_n\in Q^*$, but it is
not known which one it was, only that the lengths of both sequences agree.
Therefore, by the law of total probability,
\[
 P(v) = \sum_{q\in Q^n} P(v|q) \cdot P(q).
\]

%TODO find a reference for "Markov property" (this paragraph is currently more
%or less copied from en.wikipedia)
\begin{definition}
 A stochastic process is said to have the \emph{Markov property} if the
 conditional probability distribution of future states of the process only
 depends on the present state, not on the states before or after it.
\end{definition}

A Hidden Markov model exhibits the Markov property in two separate ways: First,
the conditional probability distribution of each state depends only on the
state directly preceding it. Second, the conditional probability distribution
of each emitted word depends only on the state that was inhabited at the time
of emission. These two conditional probability distributions are called $t$ and
$e$, and are part of the quintuple $H=(Q,V,\#,t,e)$ as defined before.

The progression of the probabilistic state machine of $H$ through the state
sequence $q=q_1\cdots q_n$ involves several separate stochastic events:
entering each state $q_1,\ldots,q_n$ in that order, then entering the state
$\#$ after $n$ other states. Therefore, by the chain rule,
\[
 P(q_1\cdots q_n) = P(q_1,\ldots,q_n,n) = P(q_1) \cdot P(q_2|q_1) \cdots P(q_n|q_1,\ldots,q_{n-1}) \cdot P(n|q_1,\ldots,q_n).
\]
The last factor, $P(n|q_1,\ldots,q_n)$ is the probability of the state sequence
having length $n$ if $q_1,\ldots,q_n$ are known or, in other words, the
probability of the state sequence terminating (by the state machine coming back
to $\#$) after these $n$ states. Since, by the first Markov property, each
state only depends on the one directly preceding it, we can reformulate each
factor in terms of the conditional probability distribution $t$:
\begin{align*}
 P(q_1) &=: t(q_1|\#), \\
 P(q_i|q_1,\ldots,q_{i-1}) &= P(q_i|q_{i-1}) =: t(q_i|q_{i-1}), \\
 P(n|q_1,\ldots,q_n) &= P(n|q_n) =: t(\#|q_n),
\end{align*}
and therefore,
\[
 P(q_1\cdots q_n) = t(q_1|\#) \cdot t(q_2|q_1) \cdots t(q_n|q_{n-1}) \cdot t(\#|q_n).
\]

In a similar way, we can rewrite $P(v|q)$ as
\[
 P(v_1\cdots v_n|q_1\cdots q_n) = \prod_{i=1}^n P(v_i|q_1\cdots q_n)
\]
and codify the second Markov property as
\[
 P(v_i|q_1\cdots q_n) = P(v_i|q_i) =: e(v_i|q_i).
\]

Putting all these results into the original equation for $P(v)$, we obtain
\[
 P(v=v_1\cdots v_n) = \sum_{q_1,\ldots,q_n} t(q_1|\#) \cdot e(v_1|q_1) \cdot \prod_{i=2}^n \mbig\brk{t(q_i|q_{i-1}) \cdot e(v_i|q_i)} \cdot t(\#|q_n).
\]

\section{Forward and backward algorithms}

When $P(v)$ is computed in this manner, the computation takes an exponential
amount of time in the sentence length $n$ since $\abs Q^n$ summands need to be
evaluated. However, for similar state sequences, some subterms can be reused
across summands, thereby reducing the required computation effort. There are
two standard schemes for this, the \emph{forward algorithm} and the
\emph{backward algorithm}.

\begin{definition}
 Let $H=(Q,V,\#,t,e)$ be an HMM, $q\in Q$ be a state, $v=v_1\cdots v_n\in V^+$
 be a sentence, and $i\in\brc{1,\ldots,n}$.\footnote{\cite{jm09} uses the
 term ``time'' and the symbol $t$ for this index. We avoid the symbol $t$
 because it is already used for the transition probability.} The \emph{forward
 weight}\footnote{The names ``forward/backward weight'' have been chosen
 deliberately, because we will see in the next section that these values
 correspond to the inside and outside weight.} $T_v(i,q)$ is the probability
 of the HMM being in state $q$ after having emitted the first $i$ words of
 $v$. The \emph{backward weight} $S_v(i,q)$ is the probability of the HMM
 generated $v$ when the first $i$ words of $v$ have already been generated and
 the HMM is in state $q$ after that many words.
\end{definition}

The forward weight can be calculated by following the same methods as in the
previous section for $P(v)$. For $i = 1$, the forward weight describes the
transition from the initial state $\#$ into $q$, and the emission of $v_1$ in
that state:
\[
 T_v(1,q) = P(v_1,q_1=q) = P(q_1=q) \cdot P(v_1|q_1=q) = t(q|\#) \cdot e(v_1|q).
\]
For $i\geq 2$, the forward weight can be calculated iteratively by first
obtaining the forward weights $T_v(i-1,q')$ for any possible previous state
$q'\in Q$, because
\[
 T_v(i,q) = P(v_1,\ldots,v_i,q_i=q) = \sum_{q'\in Q} P(v_1,\ldots,v_i,q_{i-1}=q',q_i=q)
\]
by the law of total probability, and then, by the chain rule and the Markov
properties of the HMM,
\begin{align*}
 T_v(i,q)
  &= \sum_{q'\in Q} P(v_1,\ldots,v_{i-1},q_{i-1}=q') \cdot P(q_i=q|q_{i-1}=q') \cdot P(v_i|q_i=q) \\
  &= \sum_{q'\in Q} T_v(i-1,q') \cdot t(q|q') \cdot e(v_i|q) \\
  &= e(v_i|q) \cdot \sum_{q'\in Q} T_v(i-1,q') \cdot t(q|q').
\end{align*}
The probability $P(v)$ can then be computed in a similar way as
\[
 P(v) = P(v_1,\ldots,v_n,n) = \sum_{q\in Q} P(v_1,\ldots,v_n,q_n=q) \cdot P(n|q_n=q) = \sum_{q\in Q} T_v(n,q) \cdot t(\#|q).
\]
In order to compute $P(v)$, the full matrix of $\mbig\kla{T_v(i,q)}_{i,q}$ must
be computed. Each of these forward weights can be computed in $O\mbig\kla{\abs
Q}$ time because $\abs Q$ summands need to be added. Therefore, $P(v)$ can be
computed in $O\mbig\kla{n \cdot \abs Q^2}$ time, which is much better than the
exponential time required for the initial formula for $P(v)$.

The same time complexity arises when $P(v)$ is being restated in terms of
backward weights. Backward weights can be computed in a similar manner to
forward weights, with the difference of iterating in opposite temporal order.
\begin{align*}
 P(v) &= \sum_{q\in Q} t(q|\#) \cdot e(v_1|q) \cdot S_v(1,q) \\
 \text{where } S_v(i,q) &= \begin{cases}
  t(\#|q) & \text{if }i=n, \\
  \sum_{q'\in Q} t(q'|q) \cdot e(v_{i+1}|q') \cdot S_v(i+1,q') & \text{otherwise}.
 \end{cases}
\end{align*}

\section{The Baum-Welch algorithm}

\begin{algorithm}[p!]
 \caption{Baum-Welch algorithm, based on \cite[p.~226]{jm09}. To reach a local
 maximum (or saddle point) for the corpus likelihood $p(c)$, the outermost loop
 needs to be executed until $(t,e)$ stop changing, possibly infinitely long.
 The loop condition is stated as ``not converged'' to capture that the loop is
 typically aborted once the changes to $(t,e)$ per iteration fall below some
 manually chosen threshold.\\[1em]
 The formulation of the algorithm has been altered from \cite{jm09} to also
 train the transition probabilities for the initial and final state, and to
 support a corpus with multiple sentences of different length (by taking sums
 over the time index $i$ in the E-step rather than in the M-step). The same
 alterations have already been successfully applied to an implementation of HMM
 in \cite{nel13}. \label{alg:bw-vogler}}
 \begin{algorithmic}[1]
  \algorithmheader[Input:] HMM $H_0 = (Q,V,\#,t_0,e_0)$; $V^+$-corpus $h$
  \algorithmheader[Variables:] $t:\um(Q_\#|Q_\#)$, $e:\um(V|Q)$
  \algorithmheader             $\cnt_\mathrm{tr}: Q_\#\times Q_\#\to\zr_{\geq0}$
  \algorithmheader             $\cnt_\mathrm{em}: V\times Q\to\zr_{\geq0}$
  \algorithmheader[Output:] HMM $H = (Q,V,\#,t,e)$
  \algorithmheader such that $p_H(c) \geq p_{H_0}(c)$ and $H$ maximizes $p_H(c)$ locally
  \STATE $(t,e) \leftarrow (t_0,e_0)$
  \WHILE{not converged}
   \STATE consider the HMM $(Q,V,\#,t,e)$
   \STATE $\cnt_\mathrm{tr}(q,q') \leftarrow 0$ \SUFFIXFOR{$q,q'\in Q_\#$}
   \STATE $\cnt_\mathrm{em}(w,q) \leftarrow 0$ \SUFFIXFOR{$q\in Q$ and $w\in V$}
   \FOR{$v=v_1\cdots v_n\in\operatorname{supp}(h)$}
    \STATE calculate all forward weights $T_v(i,q)$ and backward weights $S_v(i,q)$
    \FOR{$i=1,2,\ldots,n-1$}
     \FOR{$q,q'\in Q$}
      \STATE $\cnt_\mathrm{tr}(q,q') \leftarrow \cnt_\mathrm{tr}(q,q') + h(v) \cdot U_v(i,q',q)$
     \ENDFOR
    \ENDFOR
    \FOR{$i=1,2,\ldots,n$}
     \FOR{$q\in Q$}
      \STATE $\cnt_\mathrm{em}(v_i,q) \leftarrow \cnt_\mathrm{tr}(v_i,q) + h(v) \cdot R_v(i,q)$
     \ENDFOR
    \ENDFOR
    \FOR{$q\in Q$}
     \STATE $\cnt_\mathrm{tr}(q,\#) \leftarrow \cnt_\mathrm{tr}(q,\#) + h(v) \cdot R_v(1,q)$
     \STATE $\cnt_\mathrm{tr}(\#,q) \leftarrow \cnt_\mathrm{tr}(\#,q) + h(v) \cdot R_v(n,q)$
    \ENDFOR
   \ENDFOR
   \FOR{$q,q'\in Q_\#$}
    \STATE $t(q|q') \leftarrow \frac{\cnt_\mathrm{tr}(q,q')}{\sum_{q''} \cnt_\mathrm{tr}(q'',q')}$
   \ENDFOR
   \FOR{$q\in Q$ and $w\in V$}
    \STATE $e(w|q) \leftarrow \frac{\cnt_\mathrm{em}(w,q)}{\sum_{w'} \cnt_\mathrm{em}(w',q)}$
   \ENDFOR
  \ENDWHILE
 \end{algorithmic}
\end{algorithm}

The Baum-Welch algorithm is first stated in \cite{baupetsouwei70}, but since
notational conventions have changed considerably since then, we are going to
use a contemporary formulation in \cite{jm09} as a reference (see
Algorithm~\ref{alg:bw-vogler} on page~\pageref{alg:bw-vogler}).

The algorithm uses two terms that have not yet been introduced: $U_v(i,q',q)$ is
defined as the probability of the HMM progressing from state $q'$ into state
$q$ at time $i+1$ while generating the sentence $v$. $R_v(i,q)$ is the probability
of the HMM being in state $q$ at time $i$ while generating the sentence $v$.
\begin{align*}
 U_v(i,q',q) &:= P(q_i = q',q_{i+1} = q|v) && \text{for } i\in\brc{1,\ldots,\abs v-1}, q,q'\in Q \\
 R_v(i,q) &:= P(q_i=q|v) && \text{for } i\in\brc{1,\ldots,\abs v},q\in Q
\end{align*}

Both $U_v$ and $R_v$ can be expressed in terms of the forward and backward weights
$T_v$ and $S_v$, by applying the same calculation rules for probabilities that were
already used for deriving the formulas for $P(v)$, $T_v$ and $S_v$.
\begin{align*}
 R_v(i,q)
  &= P(q_i=q|v) = \frac{P(v_1,\ldots,v_n,n,q_i=q)}{P(v)} \\
  &= \frac{P(v_1,\ldots,v_i,q_i=q) \cdot P(v_{i+1},\ldots,v_n,n|q_i=q)}{P(v)} \\
  &= \frac{T_v(i,q) \cdot S_v(i,q)}{P(v)}
\end{align*}

And analogously,
\begin{align*}
 U_v(i,q',q)
  &= P(q_i = q',q_{i+1} = q|v) = \frac{P(v_1,\ldots,v_n,n,q_i=q',q_{i+1}=q)}{P(v)} \\
  &= \frac1{P(v)} \cdot \brk{\begin{matrix}
   P(v_1,\ldots,v_i,q_i=q') \cdot P(q_{i+1}=q|q_i=q') \\
   \cdot\; P(v_{i+1}|q_{i+1}=q) \cdot P(v_{i+2},\ldots,v_n,n|q_{i+1}=q)
  \end{matrix}} \\
  &= \frac{T_v(i,q') \cdot t(q|q') \cdot e(v|q) \cdot S_v(i+1,q)}{P(v)}.
\end{align*}

\section{Deriving the Baum-Welch algorithm}

\cite{jm09} describes Baum-Welch as an instance of the EM algorithm. And
indeed, the basic structure of the algorithm looks similar to the types of EM
algorithms that we have introduced in chapter 2 in several ways:
\begin{itemize}
 \item $t$ and $e$ are the model parameters that are iteratively optimized.
 \item $\cnt_\mathrm{tr}$ and $\cnt_\mathrm{em}$ act like complete-data
  corpora. They are computed using the previous model parameters, then new
  model parameters are obtained by taking the empirical probability
  distribution of these corpora, which is the efficient solution to the
  conditional maximum-likelihood estimator that is suggested by
  Lemma~\ref{lemma:empirical2}.
 \item The hidden information is the state sequence $q$ that produces the
  sentence $v$ from the corpus. It can be destructured into countable events in
  several ways (e.~g.~states only, pairs of time and states, or pairs of
  subsequent states).
 \item The way that forward weights and backward weights appear in the
  computation of $\cnt_\mathrm{tr}$ and $\cnt_\mathrm{em}$ is similar to how
  inside and outside weights appear in the computation of the inside-outside
  complete-data corpus.
\end{itemize}

Therefore, in the remainder of this chapter, we will show that the Baum-Welch
algorithm can be obtained from the generic EM algorithm from chapter~2 through
suitable instantiation of the inside-outside step mapping.

\subsection{Model parameter and countable events}

For the remainder of this section, let $H = (Q,V,\#,t,e)$ be an HMM.
Observations are sentences with words from $V$, i.~e.,
\[
 X = V^+.
\]

An IO information contains only one component which can be subject to training:
the model parameter $\omega$ which chooses the conditional probability
distribution
\[
 q_\omega \in \um_C(A|B).
\]

For a HMM, $q_\omega$ must describe both the transmission and emission
probability. We therefore choose
\[
 \Omega := \um(Q_\#|Q_\#) \times\um(V|Q)
\]
such that every model parameter $\omega=(t,e)$ is a pair of transition and
emission probability for~$H$. Moreover, we choose
\begin{align*}
 A &:= Q_\#\cup V, \\
 B &:= Q_\# \times \brc{E,T}, \\
 C &:= \mbig\brc{\mbig\kla{q',(q,T)}\mid q,q'\in Q_\#} \cup \mbig\brc{\mbig\kla{v,(q,E)}\mid v\in V, q\in Q}, \\
 q_{\omega=(t,e)}\mbig\kla{a\!\dmiddle|\!(q,b)} &:= \begin{cases}
  t(a|q) & \text{if } b = T \text{ and } a\in Q_\#, \\
  e(a|q) & \text{if } b = E \text{ and } a\in V \text{ and } q \neq\#, \\
  1 & \text{if } b = E \text{ and } a = q = \#, \\
  0 & \text{otherwise.}
 \end{cases}
\end{align*}

Herein, $E$ and $T$ are symbols such that $E,T\notin Q\cup V$ that denote if a
countable event $c\in C$ is a transition event
\[
 c = \mbig\kla{q',(q,T)} \text{ with probability } q_{(t,e)}(c) = t(q'|q).
\]
or an emission event
\[
 c = \mbig\kla{v,(q,E)} \text{ with probability } q_{(t,e)}(c) = e(v|q).
\]

The choice of $q_\omega\mbig\kla{\#\!\dmiddle|\!(\#,E)}=1$ as above ensures that
$q_\omega\in\um_C(A|B)$ for all $\omega=(t,e)$.

\begin{proof}
 Let $\omega=(t,e)\in\Omega$. It is easy to see that
 $\operatorname{supp}(q_\omega)\subseteq C$ since $q_\omega(a|b)$ was defined
 as 0 for any $c\notin C$. According to Definition~\ref{def:02-cpd}, what
 remains to be shown is that $q_\omega\mbig\kla{(q,b)}\in\um(A)$ for every
 $(q,b)\in B$. For $q\in Q_\#$ and $b = T$,
 \[
  \sum_{a\in A} q_{(t,e)}\mbig\kla{q'\!\dmiddle|\!(q,T)} = \sum_{q'\in Q_\#} q_{(t,e)}\mbig\kla{q'\!\dmiddle|\!(q,E)} = \sum_{q'\in Q_\#} t(q'|q) = 1
 \]
 because $t\in\um(Q_\#|Q_\#)$. Analogously, for $q\in Q$ and $b = E$, we have
 \[
  \sum_{a\in A} q_{(t,e)}\mbig\kla{a\!\dmiddle|\!(q,E)} = \sum_{v\in V} q_{(t,e)}\mbig\kla{v\!\dmiddle|\!(q,E)} = \sum_{v\in V} e(v|q) = 1
 \]
 because $e\in\um(V|Q)$. It remains to see that
 \[
  \sum_{a\in A} q_\omega\mbig\kla{a\!\dmiddle|\!(\#,E)} = 1,
 \]
 which is ensured by the only nonzero contribution to this sum being
 $q_\omega\mbig\kla{\#\!\dmiddle|\!(\#,E)} = 1$.
\end{proof}

It shall be noted that the countable event $\mbig\kla{\#,(\#,E)}$ is only used
to ensure $q_\omega\in\um_C(A|B)$. It does not appear in any $y\in Y$.

\subsection{Tree-shaped hidden information}

\begin{figure}[t!]
 \centering
 \begin{tikzpicture}[every node/.style={rectangle,draw}]
  \node(t0) at (0,0) { $\text{noun},(\#,T)$ };
  \node(t1) at (0,-1.5) { $\text{verb},(\text{noun},T)$ } edge[thick] (t0);
  \node(t2) at (0,-3.0) { $\text{noun},(\text{verb},T)$ } edge[thick] (t1);
  \node(t3) at (0,-4.5) { $\#,(\text{noun},T)$ }          edge[thick] (t2);
  \node(e1) at (4,-1.5) { $\text{he},(\text{noun},E)$ };
  \node(e2) at (4,-3.0) { $\text{ate},(\text{verb},E)$ };
  \node(e3) at (4,-4.5) { $\text{soup},(\text{noun},E)$ };
  \draw[thick] (t0.south east) -- (e1.north west);
  \draw[thick] (t1.south east) -- (e2.north west);
  \draw[thick] (t2.south east) -- (e3.north west);
 \end{tikzpicture}
 \caption{Example for a hidden information $y\in Y\subseteq T_C$ corresponding
 to the observation $x = \text{``He ate soup.''}$ and the state sequence
 ``noun-verb-noun''.\label{fig:03-example-y}}
\end{figure}

In order to describe hidden information $y\in Y$ as a ranked tree of countable
events from $C$, we assign a rank to each $c = \mbig\kla{a,(q,b)}\in C$ as follows:
\begin{align*}
 \rk\mbig\kla{a,(q,b)} := \begin{cases}
  0 & \text{if } b = E, \\
  2 & \text{if } b = T \text{ and } q\neq\#, \\
  0 & \text{if } b = T \text{ and } q = \#.
 \end{cases}
\end{align*}

The intuition for this choice is that each transition event causes two further
events: the emission event in the state that was entered, and the transition
event into the state after that. Emission events do not result in further
events because of the Markov property, and a transition event into the $\#$
state marks the end of the stochastic process after which no further events
occur. This rank assignment results in trees such as the one in
Figure~\ref{fig:03-example-y}.

The trees $y\in Y$ are generated by the tree grammars $H(x)$ and $K$. We define
\[
 K := \mbig\brc{Q_K, (\#,0,T), R_K} \quad\text{where}\quad Q_K = Q_\# \times \zn \times \brc{E,T}
\]
and $R_K$ contains the following rules:
\begin{align*}
 (\#,0,T) &\to \mbig\kla{q,(\#,T)}\mbig\kla{(q,1,T),(q,1,E)} &&\forall q\in Q, \\
 (q,i,T) &\to \mbig\kla{q',(q,T)}\mbig\kla{(q',i+1,T),(q',i+1,E)} &&\forall q,q'\in Q \text{ and } i\geq1, \\
 (q,i,T) &\to \mbig\kla{\#,(q,T)} &&\forall q\in Q\text{ and } i\geq 1, \\
 (q,i,E) &\to \mbig\kla{v,(q,E)} &&\forall q\in Q\text{ and } i\geq 1 \text{ and } v\in V.
\end{align*}

The IO information requires that $K$ is unambiguous. It is easy to see that
this is the case: $K$ is deterministic because each $c\in C$ is produced by at
most one rule from $R_K$, and all deterministic RTG are also unambiguous
(see~page~\pageref{lemma:02-deterministic-is-unambiguous}).






{\color{red}TODO: $H(x)$, $\pi_1$, $Y$ as $\sum_x \pi_1^{-1}(x)$; $\alpha$, $\beta$, $\chi$, $c\dangle{\omega,\mu}$, $\stepmap\mu_\mathrm{io}$}


\backmatter
% \listoffigures % TODO: necessary?
% \listofalgorithms % TODO: necessary?
\bibliographystyle{alpha}
\bibliography{citations}

\clearpage
\thispagestyle{empty}

\minisec{Erklärung}\vspace*{1.5em}

Hiermit erkläre ich, dass ich diese Arbeit im Rahmen der Betreuung im Institut
für Theoretische Informatik ohne unzulässige Hilfe Dritter verfasst und alle
Quellen als solche gekennzeichnet habe.

\vspace*{15em}

Stefan Majewsky \par
Dresden, {\color{red}August 2017} % TODO: enter final date when known

\end{document}
