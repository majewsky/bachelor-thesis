\clearpage
\section{Formulary for Chapter~3}\label{sec:appendix-formulary}
\newcommand\was[1]{\tag{p.~\pageref{#1}}}

This appendix assembles most of the formulas derived in Chapter~3. It serves as
a quick reference for the observant reader who wants to follow the derivations
without having to flip pages excessively. For most equations, the tag near the
right margin refers to the page where the equation was originally derived.

We imply that $\uh = (Q,V,\#,t,e)$ is an HMM. Let $v\in V^*$ be an observation.
\begin{align*}
 \was{eq:03-p-v}
 P(v=v_1\cdots v_n) &= \sum_{q_1,\ldots,q_n} t(q_1|\#) \cdot e(v_1|q_1) \cdot \prod_{i=2}^n \mbig\brk{t(q_i|q_{i-1}) \cdot e(v_i|q_i)} \cdot t(\#|q_n), \\
 \was{eq:03-p-v-eps}
 P(v=\eps) &= t(\#|\#).
 \intertext{In terms of the forward weight $T_v(i,q)$ for $i\in\brc{1,\ldots,n}$ and $q\in Q$:}
 \was{eq:03-p-by-t}
 P(v_1\cdots v_n) &= \sum_{q\in Q} T_v(n,q) \cdot t(\#|q), \\
 \was{eq:03-T_v}
 T_v(i,q) &= \begin{cases}
  e(v_1|q) \cdot t(q|\#) & \text{if } i = 1, \\
  e(v_i|q) \cdot \sum_{q'\in Q} T_v(i-1,q') \cdot t(q|q') & \text{if } i \in\brc{2,\ldots,n}.
 \end{cases}
 \intertext{In terms of the backward weight $S_v(i,q)$ for $i\in\brc{1,\ldots,n}$ and $q\in Q$:}
 \was{eq:03-p-by-s}
 P(v_1\cdots v_n) &= \sum_{q\in Q} t(q|\#) \cdot e(v_1|q) \cdot S_v(1,q), \\
 \was{eq:03-S_v}
 \underbrace{S_v(i,q)}_{q\in Q} &= \begin{cases}
  t(\#|q) & \text{if }i=n, \\
  \sum_{q'\in Q} t(q'|q) \cdot e(v_{i+1}|q') \cdot S_v(i+1,q') & \text{if } i\in\brc{1,\ldots,n-1}.
 \end{cases}
 \intertext{Expressions from the Baum-Welch algorithm, for $i\in\brc{1,\ldots,n}$ and $q,q'\in Q$:}
 \was{eq:03-R_v}
 R_v(i,q) &= \frac{T_v(i,q) \cdot S_v(i,q)}{P(v)}, \\
 \was{eq:03-U_v}
 U_v(i,q,q') &= \frac{T_v(i,q) \cdot t(q'|q) \cdot e(v_{i+1}|q') \cdot S_v(i+1,q')}{P(v)}.
\end{align*}

The IO information $\mu_\uh$ has $X_{\not\bot} = V^*$ and $\Omega =
\um(Q_\#|Q_\#) \times\um(V|Q)$. For any $\omega=(t,e)\in\Omega$, hidden
information is structured according to
\newcommand\twosub[2]{{\scriptstyle#1}\atop{\scriptstyle\text{\strut}#2}}
\begin{align*}
 \was{eq:03-ABC}
 A &= Q_\# \cup V, \quad B = Q_\# \times \brc T \cup Q \times \brc E, \\
 \was{eq:03-ABC}
 C &= \underbrace{\mbig\brc{\mbig\kla{q',(q,T)}\mid q,q'\in Q_\#}}_{\twosub{q_\omega(q'|(q,T)) \,=\, t(q',q)}{\text{rank 0 for $q'=\#$, 2 otherwise}}} \cup \underbrace{\mbig\brc{\mbig\kla{v,(q,E)}\mid v\in V, q\in Q}}_{\twosub{q_\omega(v|(q,E)) \,=\, e(v|q)}{\text{rank 0}}}.
\end{align*}

The grammar $K$ has states $Q_K = \brc T \times Q_\# \cup \brc E \times Q$, the initial state $(T,\#)$ and the rules
\begin{align*}
 (T,q) &\to \mbig\kla{q',(q,T)}\mbig\kla{(E,q'),(T,q')} &&\forall q\in Q_\# \text{ and } q'\in Q, \\
 (T,q) &\to \mbig\kla{\#,(q,T)} &&\forall q\in Q_\#, \\
 \was{eq:03-R_K}
 (E,q) &\to \mbig\kla{v,(q,E)} &&\forall q\in Q \text{ and } v\in V.
\end{align*}

Each $H(x)$ has states $Q_x=\brc T \times Q_\# \times \mbig\kla{\suff(x)\cup\brc\eps} \cup \brc E \times Q \times \suff(x)$, the initial state $(T,\#,x)$ and the rules
\begin{align*}
 (T,q,x') &\to \mbig\kla{q',(q,T)}\mbig\kla{(E,q',x'),(T,q',\cdr(x'))} && \forall q\in Q_\#,q'\in Q,\text{ and }x'\in\suff(x), \\
 (T,q,\eps) &\to \mbig\kla{\#,(q,T)} && \forall q\in Q_\#, \\
 \was{eq:03-R_x}
 (E,q,x') &\to \mbig\kla{\car(x'),(q,E)} && \forall q\in Q\text{ and }x'\in\suff(x).
\end{align*}

For $x=\eps$, only the initial state $(T,\#,\eps)$ is reachable and the only
nonzero inside and outside weights are
\begin{align*}
 \was{eq:03-beta-eps}
 \beta_\eps(T,\#,\eps) &= t(\#|\#) = P(\eps), \\
 \was{eq:03-alpha-eps}
 \alpha_\eps(T,\#,\eps) &= 1.
\end{align*}

The following table shows inside and outside weights for $H(x)$ with
$x=x_1\cdots x_n\in V^+$ and rule probabilities according to $q_\omega(a|b)$ of
the produced symbol $(a,b)\in C$.  Unreachable grammar states are shown greyed
out. All statements apply for all $q\in Q$ and $i\in\brc{1,\ldots,n}$ unless
otherwise noted.

\newcommand\mystrut{\rule[-0.6em]{0pt}{1.8em}}%
\newcommand\mute{\color{black!30}}
\begin{center}\begin{tabular}{|c|c||c|c|}%
 \hline\mystrut
 Grammar state $q_g$ & Reachable? & Inside weight $\beta_x(q_g)$ & Outside weight $\alpha_x(q_g)$
 \\\hhline{|=|=#=|=|}\mystrut
 \mute$(T,\#,\eps)$ & \mute no & \mute not relevant & \mute0
 \\\hline\mystrut
 $(T,\#,x)$ & yes (initial) & $P(x)$ & $1$
 \\\hline\mystrut
 \mute$(T,\#,x_i\cdots x_n)$ & \mute no & \mute not relevant & \mute0
 \\\hline\mystrut
 $(T,q,\eps)$ & yes & $t(\#|q)=S_x(n,q)$ & $T_x(n,q)$
 \\\hline\mystrut
 \mute$(T,q,x)$ & \mute no & \mute not relevant & \mute0
 \\\hline\mystrut
 $(T,q,x_i\cdots x_n)$ & for $i>1$ & $S_x(i-1,q)$ for $i>1$ & $T_x(i-1,q)$ for $i>1$
 \\\hhline{|=|=#=|=|}\mystrut
 $(E,q,x)$ & yes & $e(x_1|q)$ & $T_x(1,q) \cdot S_x(1,q) \cdot e(x_1|q)^{-1}$
 \\\hline\mystrut
 $(E,q,x_i\cdots x_n)$ & yes & $e(x_i|q)$ & $T_x(i,q) \cdot S_x(i,q) \cdot e(x_i|q)^{-1}$
 \\\hline
\end{tabular}\end{center}

\endinput
\clearpage
Let $c$ be an $V^*$-corpus. For all $q,q'\in Q$ and $v\in V$, the complete-data
corpus is defined by
\begin{align*}
 c\!\dangle{\omega,\mu_\uh}\mbig\kla{\#,(\#,T)} &= c(\eps), \\
 c\!\dangle{\omega,\mu_\uh}\mbig\kla{q,(\#,T)} &= \sum_{x\in\operatorname{supp}(h)\setminus\brc\eps} c(x) \cdot R_x(1,q), \\
 c\!\dangle{\omega,\mu_\uh}\mbig\kla{\#,(q,T)} &= \sum_{x\in\operatorname{supp}(h)\setminus\brc\eps} c(x) \cdot R_x\mbig\kla{\abs x,q}, \\
 c\!\dangle{\omega,\mu_\uh}\mbig\kla{q',(q,T)} &= \sum_{x\in\operatorname{supp}(h)\setminus\brc\eps} c(x) \cdot \sum_{i\in\brc{1,\ldots,\abs x-1}} U_x(i,q,q'), \\
 c\!\dangle{\omega,\mu_\uh}\mbig\kla{v,(q,E)} &= \sum_{x\in\operatorname{supp}(h)\setminus\brc\eps} c(x) \cdot \sum_{i\in\brc{1,\ldots,\abs x}: x_i=v} R_x(i,q), \\
\end{align*}
