\chapter{The Hidden Markov model}

The definitions in this chapter are based on \cite{vogler2015} and largely
agree with most other definitions for Hidden Markov models in contemporary
literature, e.~g.~\cite[pp.~210]{jm09}, except for variable names.

\begin{definition}
 A \emph{Hidden Markov model} (HMM) is a quintuple $H = (Q,V,\#,t,e)$ such that
 \begin{itemize}\setlength\itemsep{-0.3em}
  \item $Q$ is a non-empty alphabet (of states),
  \item $V$ is a non-empty alphabet (of letters),
  \item $\#\notin Q\cup V$ is a separate (initial and final) state,
  \item $t\in\um(Q\cup\brc{\#}|Q\cup\brc{\#})$ and $e\in\um(V|Q)$.
 \end{itemize}
\end{definition}

The Hidden Markov model describes a word as being the result of the progression
of a probabilistic state machine that starts out in $\#$, traverses states from
$Q$, and in the end reaches $\#$ again. Each time a state from $Q$ is reached,
a letter from $V$ is emitted. The sequence of all these emitted letters is the
word that is observed.

When a word $v = v_1\cdots v_n\in V^*$ is observed, is must have been caused by
a certain sequence of states $q = q_1\cdots q_n\in Q^*$, but it is not known
which one it was, only that the lengths of both sequences agree. Therefore, by
the law of total probability,
\[
 P(v) = \sum_{q\in Q^n} P(v|q) \cdot P(q).
\]

%TODO find a reference for "Markov property" (this paragraph is currently more
%or less copied from en.wikipedia)
\begin{definition}
 A stochastic process is said to have the \emph{Markov property} if the
 conditional probability distribution of future states of the process only
 depends on the present state, not on the states before or after it.
\end{definition}

A Hidden Markov model exhibits the Markov property in two separate ways: First,
the conditional probability distribution of each state depends only on the
state directly preceding it. Second, the conditional probability distribution
of each emitted letter depends only on the state that was inhabited at the time
of emission. These two conditional probability distributions are called $t$ and
$e$, and are part of the quintuple $H=(Q,V,\#,t,e)$ as defined before.

The progression of the probabilistic state machine of $H$ through the state
sequence $q=q_1\cdots q_n$ involves several separate stochastic events:
entering each state $q_1,\ldots,q_n$ in that order, then entering the state
$\#$ after $n$ other states. Therefore, by the chain rule,
\[
 P(q_1\cdots q_n) = P(q_1,\ldots,q_n,n) = P(q_1) \cdot P(q_2|q_1) \cdots P(q_n|q_1,\ldots,q_{n-1}) \cdot P(n|q_1,\ldots,q_n).
\]
The last factor, $P(n|q_1,\ldots,q_n)$ is the probability of the state sequence
having length $n$ if $q_1,\ldots,q_n$ are known or, in other words, the
probability of the state sequence terminating (by the state machine coming back
to $\#$) after these $n$ states. Since, by the Markov property, each state only
depends on the one directly preceding it, we can reformulate each factor in
terms of the conditional probability distribution $t$:
\begin{align*}
 P(q_1) &=: t(q_1|\#), \\
 P(q_i|q_1,\ldots,q_{i-1}) &= P(q_i|q_{i-1}) =: t(q_i|q_{i-1}) \\
 P(n|q_1,\ldots,q_n) &= P(n|q_n) =: t(\#|q_n),
\end{align*}
and therefore,
\[
 P(q_1\cdots q_n) = t(q_1|\#) \cdot t(q_2|q_1) \cdots t(q_n|q_{n-1}) \cdot t(\#|q_n).
\]

In a similar way, we can rewrite $P(v|q)$ as
\[
 P(v_1\cdots v_n|q_1\cdots q_n) = \prod_{i=1}^n P(v_i|q_1\cdots q_n)
\]
and codify the second Markov property as
\[
 P(v_i|q_1\cdots q_n) = P(v_i|q_i) =: e(v_i|q_i).
\]

Putting all these results into the original equation for $P(v)$, we obtain
\[
 P(v=v_1\cdots v_n) = \sum_{q_1,\ldots,q_n} t(q_1|\#) \cdot e(v_1|q_1) \cdot \prod_{i=2}^n \mbig\brk{t(q_i|q_{i-1}) \cdot e(v_i|q_i)} \cdot t(\#|q_n).
\]

\section{Forward and backward algorithms}

When $P(v)$ is computed in this manner, the computation takes an exponential
amount of time in the word length $n$ since $\abs Q^n$ summands need to be
evaluated. However, for similar state sequences, some subterms can be reused
across summands, thereby reducing the required computation effort. There are
two standard schemes for this, the \emph{forward algorithm} and the
\emph{backward algorithm}.

{\color{red}TODO}
