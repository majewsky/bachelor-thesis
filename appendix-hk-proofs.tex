\section{Proof for Lemma~\ref{lemma:03-consistent-yield}}\label{appendix:03-consistent-yield}

Let $x\in X_{\not\bot} = V^*$. We are going to show that
\[
 \mbig\brc{\pi_1(y) \dmiddle| y\in\mbig\lang{H(x)}} = \brc{x}.
\]
We shall denote the set on the left side by $\Pi_1(x)$. For $x=\eps$, we have
\[
 \mbig\lang{H(\eps)} = \mBig\brc{\mbig\kla{\#,(\#,T)}}
 \quad\text{with}\quad \Pi_1(\eps) = \mBig\brc{\pi_X\mbig\kla{\#,(\#,T)}} = \brc\eps.
\]

Otherwise, let $x = x_1\cdots x_n\in V^+$. By Definition~\ref{def:02-rtg-lang},
each $y\in\mbig\lang{H(x)}$ corresponds to at least one abstract syntax tree
$d\in D^{(T,\#,0)}\mbig\kla{H(x)}\cap T_{R_x}$, where $\pi_C(d) = y$. We can
therefore write
\[
 \Pi_1(x)
 = \mbig\brc{\pi_1\mbig\kla{\pi_C(d_0)} \!\dmiddle|\! d_0\in \hat D_x^{(T,\#,0)}}
 = \pi_1\mBig\kla{\pi_C\mbig\kla{\hat D_x^{(T,\#,0)}}},
\]
where $\hat D_x^q$ is an abbreviation for $D^q\mbig\kla{H(x)}\cap T_{R_x}$. Let
$q\in Q_x$. Each $d\in D^q\mbig\kla{H(x)}$ can have one of the following two
forms according to Definition~\ref{def:02-past}:
\begin{align*}
 d_0 &= q, &&\text{or} \\
 d_0 &= \rho(d_1,\ldots,d_k) &&\text{where~} \rho\in R_x \text{ with } \rk(\rho) = k \text{ and } d_1\in D^{q_1}\mbig\kla{H(x)}, \ldots, d_k\in D^{q_k}\mbig\kla{H(x)}.
\end{align*}
If $d\in \hat D_x^q$, the first form is not allowed because $ q\notin T_{R_x}$.
Furthermore, in the second form, the subtrees $d_1$ through $d_k$ must also be
from $T_{R_x}$. We therefore have
\[
 \hat D_x^q = \bigcup_{\rho=(q\to c(q_1,\ldots,q_k))\in R_x}
 \mbig\brc{\rho(d_1,\ldots,d_k) \!\dmiddle|\! d_1\in\hat D_x^1,\ldots, d_k\in\hat D_x^k}.
\]
For each $q\in Q_x$, this expression can be expanded by selecting the admissible
rules $\rho$. For this purpose, we assign the following names to the rules in $R_x$.
\begin{align*}
 \rho_{x,1}(q) &:= (T,\#,0) \to \mbig\kla{q,(\#,T)}\mbig\kla{(E,q,1),(T,q,1)} &&\forall q\in Q, \\
 \rho_{x,2}(q,q',i) &:= (T,q,i) \to \mbig\kla{q',(q,T)}\mbig\kla{(E,q',i+1),(T,q',i+1)} &&\forall q,q'\in Q \text{ and } i\in\brc{1,\ldots,n-1}, \\
 \rho_{x,3}(q) &:= (T,q,n) \to \mbig\kla{\#,(q,T)} &&\forall q\in Q, \\
 \rho_{x,4}(q,i) &:= (E,q,i) \to \mbig\kla{x_i,(q,E)} &&\forall q\in Q\text{ and } i\in\brc{1,\ldots,n}.
\end{align*}
We then have
\begin{align*}
 \hat D_x^{(T,\#,0)} &= \bigcup_{q'\in Q} \mbig\brc{\rho_{x,1}(q')(d_1,d_2) \!\dmiddle|\! d_1\in\hat D_x^{(E,q',1)}, d_2\in\hat D_x^{(T,q',1)}}, \\
 \hat D_x^{(T,q,i)} &= \bigcup_{q'\in Q} \mbig\brc{\rho_{x,2}(q,q',i)(d_1,d_2) \!\dmiddle|\! d_1\in\hat D_x^{(E,q',i+1)}, d_2\in\hat D_x^{(T,q',i+1)}}, \\
 \hat D_x^{(T,q,n)} &= \mbig\brc{\rho_{x,3}(q)}, \\
 \hat D_x^{(E,q,j)} &= \mbig\brc{\rho_{x,4}(q,j)}
\end{align*}
for any $q\in Q$, $i\in\brc{1,\ldots,n-1}$ and $j\in\brc{1,\ldots,n}$. Any
other $\hat D_x^q$ for $q\in Q_x$ are empty because these states are not
reachable with the available rules. We can simplify this set of equations by
inserting $\hat D_x^{(E,q,j)}$ into the first two sets of equations, yielding
\begin{align*}
 \hat D_x^{(T,\#,0)} &= \bigcup_{q'\in Q} \mbig\brc{\rho_{x,1}(q')\mbig\kla{\rho_{x,4}(q',1),d} \!\dmiddle|\! d\in\hat D_x^{(T,q',1)}}, \\
 \hat D_x^{(T,q,i)} &= \bigcup_{q'\in Q} \mbig\brc{\rho_{x,2}(q,q',i)\mbig\kla{\rho_{x,4}(q',i+1),d} \!\dmiddle|\! d\in\hat D_x^{(T,q',i+1)}}, \\
 \hat D_x^{(T,q,n)} &= \mbig\brc{\rho_{x,3}(q)}.
\end{align*}

We now apply $\pi_C$ to these abstract syntax trees. The mapping $\pi_C:T_R\to
T_C$ (introduced as $\pi_\Sigma$ on page~\pageref{def:02-pi-sigma}) replaces
each label $\rho=\mbig\kla{q\to c(q_1,\ldots,q_k)}\in R_x$ with the symbol
$c\in C$ that $\rho$ emits.
\begin{align*}
 \pi_C\mbig\kla{\hat D_x^{(T,\#,0)}} &= \bigcup_{q'\in Q} \mBig\brc{\mbig\kla{q',(\#,T)}\mBig\kla{\mbig\kla{x_1,(q',E)},\pi_C(d)} \!\dmiddle|\! d\in\hat D_x^{(T,q',1)}}, \\
 \pi_C\mbig\kla{\hat D_x^{(T,q,i)}} &= \bigcup_{q'\in Q} \mbig\brc{\mbig\kla{q',(q,T)}\mBig\kla{\mbig\kla{x_{i+1},(q',E)},\pi_C(d)} \!\dmiddle|\! d\in\hat D_x^{(T,q',i+1)}}, \\
 \pi_C\mbig\kla{\hat D_x^{(T,q,n)}} &= \mBig\brc{\mbig\kla{\#,(q,T)}}.
\end{align*}

We can now apply $\pi_X:T_C\to V^*$ (defined on page~\pageref{def:03-def-pi-x})
to extract the generated word from the trees from $T_C$ that we obtained in
this manner.
\begin{align*}
 \pi_X\mBig\kla{\pi_C\mbig\kla{\hat D_x^{(T,\#,0)}}}
 &= \bigcup_{q'\in Q} \mbigg\brc{\pi_X\mbigg\kla{\mbig\kla{q',(\#,T)}\mBig\kla{\mbig\kla{x_1,(q',E)},\pi_C(d)}} \!\dmiddle|\! d\in\hat D_x^{(T,q',1)}} \\
 &= \bigcup_{q'\in Q} \mbig\brc{\pi_X\mbig\kla{x_1,(q',E)}\pi_X\mbig\kla{\pi_C(d)} \!\dmiddle|\! d\in\hat D_x^{(T,q',1)}} \\
 &= \bigcup_{q'\in Q} \mbig\brc{x_1\pi_X\mbig\kla{\pi_C(d)} \!\dmiddle|\! d\in\hat D_x^{(T,q',1)}} \\
 &= \brc{x_1} \cdot \bigcup_{q'\in Q} \pi_X\mBig\kla{\pi_C\mbig\kla{\hat D_x^{(T,q',1)}}} \\
 \intertext{and, analogously,}
 \pi_X\mBig\kla{\pi_C\mbig\kla{\hat D_x^{(T,q,i)}}}
 &= \brc{x_{i+1}} \cdot \bigcup_{q'\in Q} \pi_X\mBig\kla{\pi_C\mbig\kla{\hat D_x^{(T,q',i+1)}}}, \\
 \pi_X\mBig\kla{\pi_C\mbig\kla{\hat D_x^{(T,q,n)}}} &= \brc\eps.
\end{align*}

The recursion in these last two equations is solved by
\[
 \pi_X\mBig\kla{\pi_C\mbig\kla{\hat D_x^{(T,q,i)}}} = \brc{x_{i+1}\cdots x_n} \quad\text{where}\quad q\in Q\text{ and } i\in\brc{1,\ldots,n-1},
\]
which leads to
\[
 \Pi_1(x) = \pi_X\mBig\kla{\pi_C\mbig\kla{\hat D_x^{(T,\#,0)}}} = \brc{x_1} \cdot \bigcup_{q'\in Q} \brc{x_2\cdots x_n} = \brc{x_1\cdots x_n} = \brc{x}.
\]

\section{Proof for Lemma~\ref{lemma:03-language-partition}}\label{appendix:03-language-partition}

The second part of the lemma follows directly from Lemma~\ref{lemma:03-consistent-yield}.
\[
 \forall x,x'\in X_{\not\bot}\colon\; x\neq x' \Rightarrow \mbig\lang{H(x)} \cap \mbig\lang{H(x')} = \emptyset
\]
because, for every $y\in\mbig\lang{H(x)}$, $\pi_1(y) = x\neq x'$ and therefore
$y\notin\mbig\lang{H(x')}$, and vice versa. It remains to be shown that
\[
 \mbig\lang K = \bigcup_{x\in X_{\not\bot}} \mbig\lang{H(x)}.
\]
% or, in other words,
% \begin{equation}
%  \label{eq:b1}
%  \forall y\in \mbig\lang K\colon \exists x\in X_{\not\bot}\colon\; y\in\mbig\lang{H(x)}
% \end{equation}
% and conversely,
% \begin{equation}
%  \label{eq:b2}
%  \forall x\in X_{\not\bot}\colon \forall y\in\mbig\lang{H(x)}\colon\; y\in\mbig\lang K.
% \end{equation}

% To prove \eqref{eq:b1}, it suffices to show that
% \begin{equation}
%  \label{eq:b3}
%  \forall y\in \mbig\lang K\colon y\in\mbig\lang{H\mbig\kla{\pi_1(y)}}.
% \end{equation}

{\color{red}TODO}
