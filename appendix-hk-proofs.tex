\section{Elided proofs from Chapter~3}\label{sec:appendix-hk-proofs}

This appendix contains the proofs for Lemmas~\ref{lemma:03-consistent-yield}
and~\ref{lemma:03-language-partition}. We imply that $\uh = (Q,V,\#,t,e)$ is an
HMM. Moreover, we imply all the definitions introduced in
Section~\ref{sect:03-deriving} to describe $\uh$ in terms of an IO information.
As a prerequisite for both proofs, we shall begin by writing down the languages
of all the grammar states of $H(x)$ and $K$.

\begin{definition}
 Let $\ug = (Q_\ug,q_0,R_\ug)$ be an RTG over $C$. For any grammar state $q\in
 Q_\ug$, the \emph{language of $q$ generated by $\ug$} is given by
 \[
  \lang\ug_q := \pi_C\mbig\kla{D^q(\ug)\cap T_{R_\ug}}.
  \qedhere
 \]
\end{definition}

The mapping $\pi_C:T_{R_\ug}(Q_\ug)\to U_C(Q_\ug)$ was introduced as $\pi_\Sigma$ on
page~\pageref{def:02-pi-sigma}. The language $\lang\ug_q$ contains all trees
from $U_C$ that can be generated by applying the grammar's rules, starting from
the state $q$. It especially holds that
\begin{equation}\label{eq:a1}
 \lang\ug = \lang\ug_{q_0}.
\end{equation}
To find $D^q(\ug)\cap T_{R_\ug}$, we first observe that, according to
Definition~\ref{def:02-past}, each $d\in D^q(\ug)$ must have one of two forms,
\[
 d = q \quad\text{or}\quad d = \rho(d_1,\ldots,d_k),
\]
where $\rho=q\to c(q_1,\ldots,q_k)\in R_\ug$ such that $\rk(\rho) = k$, and
each $d_i\in D^{q_i}(\ug)$. The first form, $d = q$, cannot contribute to
$\lang\ug_q$ since $q\notin T_{R_\ug}$. The second form is admissible, but
because $d\in T_{R_\ug}$, the subtrees $d_i$ may not have positions labeled by
a state $q\in Q_\ug$ either. We therefore have
\[
 D^q(\ug)\cap T_{R_\ug} = \bigcup_{\rho=(q\to c(q_1,\ldots,q_k))\in R_\ug}
 \mbig\brc{\rho(d_1,\ldots,d_k) \!\dmiddle|\! d_1\in D^{q_1}(\ug)\cap T_{R_\ug},\ldots, d_k\in D^{q_k}(\ug)\cap T_{R_\ug}},
\]
from which follows
\begin{equation}
 \label{eq:a2}
 \lang\ug_q = \bigcup_{\rho=(q\to c(q_1,\ldots,q_k))\in R_\ug}
 \mbig\brc{c(t_1,\ldots,t_k) \!\dmiddle|\! t_1\in\lang\ug_{q_1},\ldots, t_k\in\lang\ug_{q_k}},
\end{equation}
when $\pi_C$ is applied on both sides. We can thus describe the languages
$\mbig\lang K$ and $\mbig\lang{H(x)}$ recursively, in terms of the languages
generated from their states.

Let $x\in V^*$. Application of \eqref{eq:a2} to all reachable states in $H(x)$ yields
\begin{align*}
 \mbig\lang{H(x)}_{(T\!,q,x')} &= \bigcup_{q'\in Q}\mbig\brc{\mbig\kla{q',(q,T)}(t_1,t_2) \!\dmiddle| t_1\in\mbig\lang{H(x)}_{(E,q',x')}, t_2\in\mbig\lang{H(x)}_{(T\!,q',\cdr(x'))}}, \\
 \mbig\lang{H(x)}_{(T\!,q,\eps)} &= \mbig\brc{\mbig\kla{\#,(q,T)}} \\
 \intertext{for $q\in Q_\#$ and $x'\in\suff(x)$, and}
 \mbig\lang{H(x)}_{(E,q,x')} &= \mbig\brc{\mbig\kla{\car(x'),(q,E)}}
\end{align*}
for $q\in Q$ and $x'\in\suff(x)$. Inserting the last equation into the one above it, we can describe $\mbig\lang{H(x)} = \mbig\lang{H(x)}_{(T\!,\#,\eps)}$ recursively by
\begin{subequations}\label{eq:a3}\begin{align}
 \mbig\lang{H(x)}_{(T\!,q,x')} &= \bigcup_{q'\in Q}\mBig\brc{\mbig\kla{q',(q,T)}\mBig\kla{\mbig\kla{\car(x'),(q',E)},t} \!\dmiddle| t\in\mbig\lang{H(x)}_{(T\!,q',\cdr(x'))}}, \label{eq:a3a} \\
 \mbig\lang{H(x)}_{(T\!,q,\eps)} &= \mbig\brc{\mbig\kla{\#,(q,T)}}. \label{eq:a3b}
\end{align}\end{subequations}

Analogous application of~\eqref{eq:a2} to all states in $K$, followed by
insertion of the expression for $\lang K_{(E,q)}$ into that of $\lang
K_{(T\!,q)}$, results in
\begin{equation}\label{eq:a4}
 \lang K_{(T\!,q)} = \mbig\brc{\mbig\kla{\#,(q,T)}} \cup \bigcup_{q'\in Q}\bigcup_{v\in V}\mBig\brc{\mbig\kla{q',(q,T)}\mBig\kla{\mbig\kla{v,(q',E)},t} \!\dmiddle| t\in\lang K_{(T\!,q')}}
\end{equation}
for any $q\in Q_\#$. An important observation can be made from the structure of~\eqref{eq:a3}.
\begin{equation}\label{eq:a5}
 \forall x\in V^+, q\in Q\colon\;
 \mbig\lang{H(x)}_{(T,q,\cdr(x))} = \mbig\lang{H\mbig\kla{\cdr(x)}}_{(T,q,\cdr(x))}.
\end{equation}
This equivalence holds because $Q_{\cdr(x)}$ and $R_{\cdr(x)}$ are strict
subsets of $Q_x$ and $R_x$, respectively, and furthermore, because $R_x$ does
not contain any rules not in $R_{\cdr(x)}$ that expand grammar states from
$Q_{\cdr(x)}$. Therefore, we have
\[
 D^q\mbig\kla{H(x)} = D^q\mBig\kla{H\mbig\kla{\cdr(x)}} \quad\forall q\in Q_{\cdr(x)},
\]
from which follows~\eqref{eq:a5}.

\subsection{Proof of Lemma~\ref{lemma:03-consistent-yield}}

Let $x\in V^*$. Lemma~\ref{lemma:03-consistent-yield} is equivalent to
\[
 \pi_X\mbig\kla{\mbig\lang{H(x)}} = \brc x,
\]
since $\pi_X$ only differs in domain from $\pi_1$. Applying $\pi_X$
to the equations~\eqref{eq:a3} yields
\begin{align*}
 \pi_X\mbig\kla{\mbig\lang{H(x)}_{(T\!,q,x')}} &= \bigcup_{q'\in Q}\mBig\brc{\pi_X\mbig\kla{\car(x'),(q',E)}\pi_X(d) \!\dmiddle| d\in\mbig\lang{H(x)}_{(T\!,q',\cdr(x'))}} \\
 &= \mbig\brc{\car(x')} \cdot \bigcup_{q'\in Q} \pi_X\mbig\kla{\mbig\lang{H(x)}_{(T\!,q',\cdr(x'))}}, \\
 \pi_X\mbig\kla{\mbig\lang{H(x)}_{(T\!,q,\eps)}} &= \brc\eps.
\end{align*}

This recursive definition of $\pi_X\mbig\kla{\mbig\lang{H(x)}_{(T\!,q,x')}}$, for
any $x'\in\operatorname{suff}(x)\cup\brc\eps$, does not depend on $q$. We can therefore
abbreviate $\pi_X\mbig\kla{\mbig\lang{H(x)}_{(T\!,q,x')}}$ as $\Pi_x(x')$ and
obtain
\[
 \Pi_x(x') = \begin{cases}
  \mbig\brc{\car(x')} \cdot \Pi_x\mbig\kla{\cdr(x')} & \text{if } x'\neq\eps, \\
  \brc\eps & \text{if } x' = \eps,
 \end{cases}
\]
which is clearly satisfied by $\Pi_x(x') = \brc{x'}$. Therefore, we finally have
\[
 \pi_X\mbig\kla{\mbig\lang{H(x)}} = \pi_X\mbig\kla{\mbig\lang{H(x)}_{(T\!,\#,x)}} = \Pi_x(x) = \brc x.
\]

\subsection{Proof of Lemma~\ref{lemma:03-language-partition}}

\begin{repeatlemma}{\ref{lemma:03-language-partition}}
 With the definitions from Section~\ref{sect:03-deriving},
 \[
  \mbig\lang K = \bigcup_{x\in X_{\not\bot}} \mbig\lang{H(x)}.
 \]
\end{repeatlemma}

Let $\pi_Q: T_C\to Q_\#$ be the mapping defined by
\[
 \pi_Q(y) := q \text{ such that } y(\eps) = \mbig\kla{a,(q,b)}.
\]
That is, $\pi_Q$ extracts the state $q$ from the label $\mbig\kla{a,(q,b)}$
of the root position of any $y$. We are going to show that
\begin{equation}\label{eq:p21}
 \forall y\in T_C\colon\;
 \mBig\kla{y\in\mbig\lang K_{(T\!,\pi_Q(y))}}
 \Leftrightarrow
 \mBig\kla{y\in\mbig\lang{H\mbig\kla{\pi_X(y)}}_{(T\!,\pi_Q(y),\pi_X(y))}},
\end{equation}
by structural induction on $y$. According to the definition of the ranked
alphabet $C$, each $y\in T_C$ can have one of four forms,
\begin{align*}
 y_1 &:= \mbig\kla{\#,(\#,E)}, \\
 y_2(v,q) &:= \mbig\kla{v,(q,E)} &&\text{ where } (v,q) \in (V\times Q) \cup \mbig\brc{(\#,\#)}, \\
 y_3(q) &:= \mbig\kla{\#,(q,T)} &&\text{ where } q\in Q_\#, \\
 y_4(q,q',y_e,y_t) &:= \mbig\kla{q',(q,T)}(y_e,y_t) &&\text{ where } q\in Q_\#, q'\in Q, \text{ and } y_e,y_t\in T_C.
\end{align*}
The first three forms are the base cases for the structural induction on $y$,
and the last form represents the inductive step.

\subsubsection*{Base cases}

For $y=y_1$, we have $\pi_Q(y) = \#$, $\pi_X(y) = \eps$, and \eqref{eq:p21} holds
because
\[
 \mbig\kla{\#,(\#,E)} \notin \mbig\lang K_{(T\!,\#)}
 \quad\text{ and }\quad
 \mbig\kla{\#,(\#,E)} \notin \mbig\lang{H(\eps)}_{(T\!,\#,\eps)}
\]
according to~\eqref{eq:a3} and~\eqref{eq:a4}. For any $y=y_2(v,q)$, we have
$\pi_Q(y)=q$, $\pi_X(y)=v$, and analogously,
\[
 \mbig\kla{v,(q,E)} \notin \mbig\lang K_{(T\!,q)}
 \quad\text{ and }\quad
 \mbig\kla{v,(q,E)} \notin \mbig\lang{H(v)}_{(T\!,q,v)}.
\]
For any $y=y_3(q)$, we have $\pi_Q(y) = q$ and $\pi_X(y)=\eps$, and \eqref{eq:p21} is
satisfied because both
\[
 \mbig\kla{\#,(q,T)} \in \mbig\lang K_{(T\!,q)}
 \quad\text{ and }\quad
 \mbig\kla{\#,(q,T)} \in \mbig\lang{H(\eps)}_{(T\!,q,\eps)}
\]
hold if (and only if) $q=\#$.

\subsubsection*{Induction step: ``$\Rightarrow$'' direction}

Consider any
\[
 y = y_4(q,q',y_e,y_t) = \mbig\kla{q',(q,T)}(y_e,y_t).
\]
Since $\pi_Q(y) = q$, we see from~\eqref{eq:a4} that $y\in\mbig\lang K_{(T\!,q)}$ precisely for
\[
 y_t \in \mbig\lang K_{(T\!,q')}
 \quad\text{ and }\quad
 y_e = \mbig\kla{v,(q',E)}
 \quad\text{ where }\quad
 v\in V.
\]
From this follows $\pi_X(y) = v\pi_X(y_t)$ and, therefore,
\[
 \car\mbig\kla{\pi_X(y)} = v
 \quad\text{ and }\quad
 \cdr\mbig\kla{\pi_X(y)} = \pi_X(y_t).
\]
Since $y_t$ is a substructure of $y$, it can be assumed to
satisfy~\eqref{eq:p21} (\emph{induction hypothesis}). Since $y_t\in\mbig\lang
K_{(T\!,q')}$, we can see from~\eqref{eq:a4} that only $\pi_Q(y_t) = q'$ is
possible. Therefore, by the induction hypothesis,
\[
 y_t \in \mbig\lang{H\mbig\kla{\pi_X(y_t)}}_{(T\!,q',\pi_X(y_t))}.
\]
Because of~\eqref{eq:a5}, we can replace $\pi_X(y_t) = \cdr\mbig\kla{\pi_X(y)}$ by $\pi_X(y)$ to obtain
\[
 y_t \in \mbig\lang{H\mbig\kla{\pi_X(y)}}_{(T\!,q',\pi_X(y_t))}.
\]
We therefore have
\[
 y = \mbig\kla{q',(q,T)}\mBig\kla{\mbig\kla{\car\mbig\kla{\pi_X(y)},(q',E)},y_t}
 \quad\text{ with }\quad
 \begin{cases}
  &\hspace*{-1em} q \in Q, q' \in Q_\#, \text{ and}\\
  &\hspace*{-1em} y_t \in \mbig\lang{H\mbig\kla{\pi_X(y)}}_{(T\!,q',\cdr(\pi_X(y)))}.
 \end{cases}
\]
from which follows, according to~\eqref{eq:a3a}, that
\[
 y \in \mbig\lang{H\mbig\kla{\pi_X(y)}}_{(T\!,q,\pi_X(y))}.
\]

\subsubsection*{Induction step: ``$\Leftarrow$'' direction}

Consider any $y = y_4(q,q',y_e,y_t)$. Let $x := \pi_X(y)$. Since $\pi_Q(y) = q$,
it follows from~\eqref{eq:a3} that
$y\in\mbig\lang{H(x)}_{(T\!,\pi_Q(y),x)}$ if and only if
\[
 y_e = \mbig\kla{\car(x),(q',E)}
 \quad\text{ and }\quad
 y_t \in \mbig\lang{H(x)}_{(T\!,q',\cdr(x))}.
\]
Because of~\eqref{eq:a5}, the second statement is equivalent to
\[
 y_t \in \mbig\lang{H\mbig\kla{\cdr(x)}}_{(T\!,q',\cdr(x))}.
\]
From~\eqref{eq:a3}, we see that $\pi_Q(y_t)$ must be $q'$. Since $y_t$ is a
substructure of $y$, it can be assumed to satisfy~\eqref{eq:p21}, from which
follows that $y_t \in \mbig\lang K_{(T\!,q')}$. According to~\eqref{eq:a4}, we
therefore have $y\in\mbig\lang K_{(T\!,q)}$ for any
$y\in\mbig\lang{H(x)}_{(T\!,\pi_Q(y),x)}$. The proof for~\eqref{eq:p21} is
thereby complete.

\subsubsection*{Conclusion of the proof}

From~\eqref{eq:p21} follows Lemma~\ref{lemma:03-language-partition} because
\begin{align*}
 \mbig\lang K
 &= \mbig\lang K_{(T\!,\#)} \\
 &\overset{(*)}= \mbig\brc{y\in \pi_Q^{-1}(\#) \!\dmiddle|\! y\in\mbig\lang K_{(T\!,\#)}} \\
 &= \mbig\brc{y\in \pi_Q^{-1}(\#) \!\dmiddle|\! y\in\mbig\lang K_{(T\!,\pi_Q(y))}} \\
 &\hspace*{-0.4em}\overset{\eqref{eq:p21}}= \mbig\brc{y\in \pi_Q^{-1}(\#) \!\dmiddle|\! y\in\mbig\lang{H\mbig\kla{\pi_X(y)}}_{(T\!,\pi_Q(y),\pi_X(y))}} \\
 &= \mbig\brc{y\in \pi_Q^{-1}(\#) \!\dmiddle|\! y\in\mbig\lang{H\mbig\kla{\pi_X(y)}}_{(T\!,\#,\pi_X(y))}} \\
 &= \bigcup_{x\in X_{\not\bot}} \mbig\brc{y\in \pi_Q^{-1}(\#) \!\dmiddle|\! y\in\mbig\lang{H\mbig\kla{\pi_X(y)}}_{(T\!,\#,\pi_X(y))}, x = \pi_X(y)} \\
 &= \bigcup_{x\in X_{\not\bot}} \mbig\brc{y\in \pi_Q^{-1}(\#) \!\dmiddle|\! y\in\mbig\lang{H(x)}_{(T\!,\#,x)}, x = \pi_X(y)} \\
 &\overset{(*)}= \bigcup_{x\in X_{\not\bot}} \mbig\brc{y\in \mbig\lang{H(x)}_{(T\!,\#,x)} \!\dmiddle|\! x = \pi_X(y)} \\
 &= \bigcup_{x\in X_{\not\bot}} \mbig\brc{y\in \mbig\lang{H(x)} \!\dmiddle|\! x = \pi_X(y)} \\
 &= \bigcup_{x\in X_{\not\bot}} \mbig\lang{H(x)}
\end{align*}
because of Lemma~\ref{lemma:03-consistent-yield}. At the two places marked $(*)$, we used that
\[
 \pi_Q\mbig\kla{\mbig\lang K_{(T\!,\#)}} = \brc\#
 \quad\text{ and }\quad
 \pi_Q\mbig\kla{\mbig\lang{H(x)}_{(T\!,\#,x)}} = \brc\#
\]
for any $x\in X_{\not\bot}$, which follows from~\eqref{eq:a4} and~\eqref{eq:a3}, respectively.
