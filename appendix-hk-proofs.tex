\section{Elided proofs from Chapter~3}\label{sec:appendix-hk-proofs}

This appendix contains the proofs for Lemmas~\ref{lemma:03-consistent-yield}
and~\ref{lemma:03-language-partition}. We imply that $\uh = (Q,V,\#,t,e)$ is an
HMM. Moreover, we imply all the definitions introduced in
Section~\ref{sect:03-deriving} to describe $\uh$ in terms of an IO information.
As a prerequisite for both proofs, we shall begin by writing down the languages
of all the grammar states of $H(x)$ and $K$.

\begin{definition}
 Let $\ug = (Q_\ug,q_0,R_\ug)$ be an RTG over $C$. For any grammar state $q\in
 Q_\ug$, the \emph{language of $q$ under $\ug$} is given by
 \[
  \lang\ug_q := \pi_C\mbig\kla{D^q(\ug)\cap T_{R_\ug}}.
  \qedhere
 \]
\end{definition}

The mapping $\pi_C:T_{R_\ug}(Q_\ug)\to U_C(Q_\ug)$ was introduced as $\pi_\Sigma$ on
page~\pageref{def:02-pi-sigma}. The language $\lang\ug_q$ contains all trees
from $U_C$ that can be generated by applying the grammar's rules, starting from
the state $q$. It especially holds that
\begin{equation}\label{eq:a1}
 \lang\ug = \lang\ug_{q_0}.
\end{equation}
To find $D^q(\ug)\cap T_{R_\ug}$, we first observe that, according to
Definition~\ref{def:02-past}, each $d\in D^q(\ug)$ must have one of two forms,
\[
 d = q \quad\text{or}\quad d = \rho(d_1,\ldots,d_k),
\]
where $\rho=q\to c(q_1,\ldots,q_k)\in R_\ug$ such that $\rk(\rho) = k$, and
each $d_i\in D^{q_i}(\ug)$. The first form, $d = q$, cannot contribute to
$\lang\ug_q$ since $q\notin T_{R_\ug}$. The second form is admissible, but
because $d\in T_{R_\ug}$, the subtrees $d_i$ may not contain any states $q\in
Q_\ug$ either. We therefore have
\[
 D^q(\ug)\cap T_{R_\ug} = \bigcup_{\rho=(q\to c(q_1,\ldots,q_k))\in R_\ug}
 \mbig\brc{\rho(d_1,\ldots,d_k) \!\dmiddle|\! d_1\in D^{q_1}(\ug)\cap T_{R_\ug},\ldots, d_k\in D^{q_k}(\ug)\cap T_{R_\ug}},
\]
from which follows
\begin{equation}
 \label{eq:a2}
 \lang\ug_q = \bigcup_{\rho=(q\to c(q_1,\ldots,q_k))\in R_\ug}
 \mbig\brc{c(t_1,\ldots,t_k) \!\dmiddle|\! t_1\in\lang\ug_{q_1},\ldots, t_k\in\lang\ug_{q_k}},
\end{equation}
when $\pi_C$ is applied on both sides. We can thus describe the languages
$\mbig\lang K$ and $\mbig\lang{H(x)}$ recursively, in terms of the languages of
their grammar states.

Let $x\in V^*$. Application of \eqref{eq:a2} to all reachable states in $H(x)$ yields
\begin{align*}
 \mbig\lang{H(x)}_{(T,q,x')} &= \bigcup_{q'\in Q}\mbig\brc{\mbig\kla{q',(q,T)}(t_1,t_2) \!\dmiddle| t_1\in\mbig\lang{H(x)}_{(E,q',x')}, t_2\in\mbig\lang{H(x)}_{(T,q',\cdr(x'))}}, \\
 \mbig\lang{H(x)}_{(T,q,\eps)} &= \mbig\brc{\mbig\kla{\#,(q,T)}} \\
 \intertext{for $q\in Q_\#$ and $x'\in\suff(v)$, and}
 \mbig\lang{H(x)}_{(E,q,x')} &= \mbig\brc{\mbig\kla{\car(x'),(q,E)}}
\end{align*}
for $q\in Q$ and $x'\in\suff(x)$. Inserting the last equation into the one above it, we can describe $\mbig\lang{H(x)} = \mbig\lang{H(x)}_{(T,\#,\eps)}$ recursively by
\begin{subequations}\label{eq:a3}\begin{align}
 \mbig\lang{H(x)}_{(T,q,x')} &= \bigcup_{q'\in Q}\mBig\brc{\mbig\kla{q',(q,T)}\mBig\kla{\mbig\kla{\car(x'),(q',E)},t} \!\dmiddle| t\in\mbig\lang{H(x)}_{(T,q',\cdr(x'))}}, \\
 \mbig\lang{H(x)}_{(T,q,\eps)} &= \mbig\brc{\mbig\kla{\#,(q,T)}}.
\end{align}\end{subequations}

Analogous application of~\eqref{eq:a2} to all states in $K$, followed by
insertion of the expression for $\lang K_{(E,q)}$ into that of $\lang
K_{(T,q)}$, results in
\begin{equation}\label{eq:a4}
 \lang K_{(T,q)} = \mbig\brc{\mbig\kla{\#,(q,T)}} \cup \bigcup_{q'\in Q}\bigcup_{v\in V}\mBig\brc{\mbig\kla{q',(q,T)}\mBig\kla{\mbig\kla{v,(q',E)},t} \!\dmiddle| t\in\lang K_{(T,q')}}.
\end{equation}

\subsection{Proof for Lemma~\ref{lemma:03-consistent-yield}}

Let $x\in V^*$. Lemma~\ref{lemma:03-consistent-yield} is equivalent to
\[
 \pi_X\mbig\kla{\mbig\lang{H(x)}} = \brc x,
\]
since $\pi_X$ only differs in domain from $\pi_1$. Applying $\pi_X$
to the equations~\eqref{eq:a3} yields
\begin{align*}
 \pi_X\mbig\kla{\mbig\lang{H(x)}_{(T,q,vx')}} &= \bigcup_{q'\in Q}\mBig\brc{\pi_X\mbig\kla{v,(q',E)}\pi_X(d) \!\dmiddle| d\in\mbig\lang{H(x)}_{(T,q',x')}} \\
 &= \brc v \cdot \bigcup_{q'\in Q} \pi_X\mbig\kla{\mbig\lang{H(x)}_{(T,q',x')}}, \\
 \pi_X\mbig\kla{\mbig\lang{H(x)}_{(T,q,\eps)}} &= \brc\eps.
\end{align*}

This recursive definition of $\pi_X\mbig\kla{\mbig\lang{H(x)}_{(T,q,x')}}$, for
any $x'\in\operatorname{suff}(x)$, does not depend on $q$. We can therefore
abbreviate $\pi_X\mbig\kla{\mbig\lang{H(x)}_{(T,q,x')}}$ as $\Pi_x(x')$ and
obtain
\[
 \Pi_x(vx') = \brc v \cdot \Pi_x(x')
 \quad\text{and}\quad
 \Pi_x(\eps) = \brc\eps,
\]
which is clearly satisfied by $\Pi_x(x') = \brc{x'}$. Therefore, we finally have
\[
 \pi_X\mbig\kla{\mbig\lang{H(x)}} = \pi_X\mbig\kla{\mbig\lang{H(x)}_{(T,\#,x)}} = \Pi_x(x) = \brc x.
\]

\subsection{Proof for Lemma~\ref{lemma:03-language-partition}}

We define
\[
 L(q) := \bigcup_{x\in X_{\not\bot}} \mbig\lang{H(x)}_{(T,q,x)}
\]
for all $q\in Q_\#$. Inserting~\eqref{eq:a3} into the right side, we obtain
\begin{align*}
 L(q)
 &= \mbig\lang{H(\eps)}_{(T,q,\eps)} \cup \bigcup_{x\in V^+} \mbig\lang{H(x)}_{(T,q,x)} \\
 &= \mbig\brc{\mbig\kla{\#,(q,T)}} \cup \bigcup_{x\in V^+} \bigcup_{q'\in Q}\mBig\brc{\mbig\kla{q',(q,T)}\mBig\kla{\mbig\kla{\car(x),(q',E)},t} \!\dmiddle| t\in\mbig\lang{H(x)}_{(T,q',\cdr(x))}}.
\end{align*}

We can express $x\in V^+$ as $x = vx'$ with $v\in V$ and $x\in V^*$. Therefore,
\begin{align*}
 L(q)
 &= \mbig\brc{\mbig\kla{\#,(q,T)}} \cup \bigcup_{q'\in Q} \bigcup_{v\in V} \bigcup_{x'\in V^*} \mBig\brc{\mbig\kla{q',(q,T)}\mBig\kla{\mbig\kla{v,(q',E)},t} \!\dmiddle| t\in\mbig\lang{H(x)}_{(T,q',x')}} \\
 &= \mbig\brc{\mbig\kla{\#,(q,T)}} \cup \bigcup_{q'\in Q} \bigcup_{v\in V} \mBig\brc{\mbig\kla{q',(q,T)}\mBig\kla{\mbig\kla{v,(q',E)},t} \!\dmiddle| t\in\underbrace{\bigcup_{x'\in V^*} \mbig\lang{H(x)}_{(T,q',x')}}_{=L(q')}}.
\end{align*}

This recursive expression for $L(q)$ is the same as for $\lang K_{(T,q)}$ in~\eqref{eq:a4}. Therefore,
\[
 \lang K_{(T,q)} = L(q)
\]
for all $q\in Q_\#$ and, finally, Lemma~\ref{lemma:03-language-partition} emerges because
\[
 \lang K = \lang K_{(T,\#)} = L(\#)
 = \bigcup_{x\in X_{\not\bot}} \mbig\lang{H(x)}_{(T,\#,x)}
 = \bigcup_{x\in X_{\not\bot}} \mbig\lang{H(x)}.
\]
